\documentclass[11pt, a4paper,polish,twoside]{report}
\usepackage{graphicx}
\usepackage{times}
\usepackage[T1]{fontenc}
\usepackage[polish]{babel}
\usepackage[utf8]{inputenc}
\usepackage{lmodern}
\usepackage{hyperref}
\usepackage{graphicx}
\usepackage{pdfpages}
\usepackage{mathtools}
\selectlanguage{polish}
\usepackage{indentfirst}
\usepackage{gensymb}
\usepackage{rotating}
\widowpenalty=10000
\pagestyle {plain}
\usepackage{float}
\usepackage{geometry}
\usepackage{mathptmx}
\newgeometry{tmargin=2.5cm, bmargin=3.0cm, left=3.5cm, right=2.5cm}
\setcounter{secnumdepth}{2}
\linespread{1.3}
\usepackage{etoolbox}
\patchcmd{\chapter}{\thispagestyle{plain}}{\thispagestyle{fancy}}{}{}
\setlength{\headsep}{1.5cm}
\usepackage{listings}
\usepackage{lastpage}
\usepackage[MeX]{polski}
\graphicspath{ {./pictures/} }
\newcommand{\HRule}{\rule{\linewidth}{0.5mm}}
\usepackage{fancyhdr}
\pagestyle{fancy}
\setcounter{tocdepth}{1}
\usepackage[backend=bibtex]{biblatex}
\usepackage{siunitx}
\bibliography{bibliografia.bib}

\lstdefinelanguage{JavaScript}{
  keywords={typeof, new, true, false, catch, function, return, null, catch, switch, var, if, in, while, do, else, case, break},
  keywordstyle=\color{blue}\bfseries,
  ndkeywords={class, export, boolean, throw, implements, import, this},
  ndkeywordstyle=\color{darkgray}\bfseries,
  identifierstyle=\color{black},
  sensitive=false,
  comment=[l]{//},
  morecomment=[s]{/*}{*/},
  commentstyle=\color{purple}\ttfamily,
  stringstyle=\color{red}\ttfamily,
  morestring=[b]',
  morestring=[b]"
}


\begin{document}

%%%%%%%%%%%%%%%%%%%%%%%%		STRONA TYTUŁOWA

\begin{titlepage}
	\begin{center}
		\textsc{\LARGE Politechnika Poznańska}\\[0.3cm] 
		\textsc{\large Wydział Informatyki}\\[0.3cm]
		\begin{figure}[!ht]
		\centering
		\includegraphics[scale=0.3]{pictures/logoPP.png}
		\end{figure}
		\textsc{\Large Praca magisterska}\\[0.5cm]
		\HRule \\[0.4cm]
		{ \huge  System bezpieczeństwa i zarządzania domem w IoT\\[0.4cm] }
		\HRule \\[2.5cm]
		\noindent
		\begin{minipage}{0.4\textwidth}
			\begin{flushleft} 
				\large \emph{Autor:}\\ inż. 
				\\Mateusz Szymkowiak
			\end{flushleft}
		\end{minipage}%
		\begin{minipage}{0.4\textwidth}
			\begin{flushright} \large
				\emph{Promotor:} \\ \hfill dr inż. 
				\\Michał Melosik
				
			\end{flushright}
		\end{minipage}
		
		\vfill
		
		% Bottom of the page
		{\large \today}
		
		
	\end{center}

\end{titlepage}



\lhead{\scriptsize \textsc{Praca dyplomowa pt. ,,System bezpieczeństwa i zarządzania domem w IoT
''\\
		Politechnika Poznańska, Wydział informatyki}}
\rhead{{\thepage{} z \pageref{LastPage}}}



%%%%%%%%%%%%%%%%%%%%%%%%		
\newpage
\null
\thispagestyle{empty}
\newpage
%\includepdf[pages={1}]{./pictures/karta_pracy.pdf}

%%%%%%%%%%%%%%%%%%%%%%%%		STRESZCZENIE
\vspace*{\fill}
\begin{center}
{\centering \Huge \bfseries Streszczenie}\\
\end{center}
W pracy zaprezentowano system bezpieczeństwa i zarządzania domem w Internet of Things. Głównymi obiektami badań, na którym skupiono się w tej pracy są systemy detekcji oraz rozpoznawania twarzy. Na potrzeby tej pracy utworzono aplikację internetową umożliwiającą sprawne testowanie wybranych algorytmów, sieci neuronowych oraz usług pozwalających na analizę obrazu. Z powodu niewystarczającej mocy obliczeniowej platformy Raspberry Pi, przetestowano środowiska Azure oraz AWS pozwalające na uruchomienie aplikacji w chmurze.
\\
\\
\\
\begin{center}
 {\Huge \bfseries  Abstract}\\
\end{center}
english
\\
\\
\\
\\
\\
\\
\\
\\
\\
\vspace*{\fill}
%%%%%%%%%%%%%%%%%%%%%%%%		SPIS TREŚCI

\tableofcontents

%%%%%%%%%%%%%%%%%%%%%%%%	WSTĘP
\chapter{Wstęp}
Internet rzeczy (Internet of Things -IoT) to koncepcja przedstawiająca sieć urządzeń fizycznych połączonych inteligentną siecią KNX lub internetową, komunikujących się między sobą i wymieniających dane. Podstawowym celem IoT jest stworzenie inteligentnych przestrzeni- miast, budynków lub systemów związanych z życiem codziennym. Jednym z zastosowań takich systemów są inteligentne domy (smart homes), wykorzystujące czujniki oraz aktuatory do zarządzania domem.

Na potrzeby tej pracy opracowano system bezpieczeństwa oparty o system detekcji i rozpoznawania twarzy na obrazie.

%TODO dodać coś o przetwarzani obrazu, rozpoznawaniu twarzy itd

W rozdziale ,,Cel i zakres pracy'' przedstawiono główne założenia projektu oraz zakres prac autora.

W rozdziale ,,Wstęp teoretyczny'' omówiono podstawowe zagadnienia związane z treścią tej pracy magisterskiej.

W rozdziale ,,Przegląd dostępnych metod rozpoznawania twarzy'' porównano możliwości kilku z dostępnych usług lub oprogramowania pozwalającego na rozpoznawanie twarzy.

W rozdziale ,,System zarządzania metodami rozpoznawania twarzy'' przedstawiono strukturę stworzonej aplikacji, jej podział na moduły oraz wybrane dla każdego z nich środowisko uruchomieniowe.

W rozdziale ,,Aplikacja internetowa'' zaprezentowano wszystkie strony utworzone na potrzeby projektu.

W rozdziale ,,Aplikacja konsolowa'' omówiono sposób wykorzystania wcześniej opisanych usług oraz algorytmy odpowiedzialne za działanie systemu do testowania różnych rozwiązań dla problemu identyfikacji twarzy.

W rozdziale ,,Aplikacja konsolowa do zarządzania domem'' opisano sposób wykorzystania czujników oraz kamery podłączonej bezpośrednio do platformy Raspberry Pi.

W rozdziale ,,Porównanie wykorzystanych technologi'' zbadano oraz porównano skuteczność działania technologi wybranych na potrzeby tej pracy.

W rozdziale ,,Podsumowanie'' zawarto podsumowanie zgromadzonych informacji oraz przedstawiono wnioski wynikające z przeprowadzonych badań.



%TODO dodać opisy kolejnych rozdziałów


%%%%%%%%%%%%%%%%%%%%%%%%	CEL I ZAKRES PRAC
\chapter{Cel i zakres prac}
Celem pracy było stworzenie systemu bezpieczeństwa i zarządzania domem w IoT. Głównymi założeniami było opracowanie systemu pozwalającego na porównanie wybranych metod i usług pozwalających na analizę obrazu oraz kontrolę stanu czujników podłączonych do systemu. IoT jest bardzo prężnie rozwijającą się ideą, dlatego postanowiono porównać wybrane technologie i usługi ułatwiające zarządzanie oraz integrację z takimi aplikacjami.

Jako platformę sprzętową wybrano bardzo popularne w zastosowaniach IoT urządzenie Raspberry Pi 3. Z powodu jego ograniczonych zasobów obliczeniowych zintegrowano chmurowe usługi Azure oraz AWS. Zakres pracy obejmuje następujące zagadnienia:
\begin{itemize}
\item dobór odpowiedniej platformy sprzętowej oraz oprogramowania,
\item porównanie działania wybranych algorytmów detekcji oraz rozpoznawania twarzy,
\item wybór i porównanie wybranych technologi umożliwiających integrację z systemami IoT,
\item wykrywanie zdarzeń rejestrowanych przez wybrane czujniki
%TODO is it really needed?
\item opracowanie systemu przesyłania powiadomień
\item porównanie przydatności zintegrowanych usług w zastosowaniu dla systemu bezpieczeństwa domu
\end{itemize}

%%%%%%%%%%%%%%%%%%%%%%%%	Wstęp teoretyczny
\chapter{Wstęp teoretyczny}
\section{Budowa aplikacji}
\begin{figure}[H]
	\centering
	\includegraphics[scale=0.8]{schemat_systemu.png}
	\caption{Budowa systemu zarządzania domem}
	\label{fig:schemat_systemu}
\end{figure}
Aplikację powstałą na potrzeby tej pracy można podzielić na 3 moduły, które zostały przedstawione na rysunku \ref{fig:schemat_systemu}, a składają się na nie :
\begin{itemize}
\item Aplikacja webowa- interfejs pozwalający na zlecanie nowych zadań aplikacji konsolowej oraz odczyt wyników przesłanych przez nią oraz przez program zarządzający domem,
\item Aplikacja konsolowa- przetwarza zadania detekcji oraz rozpoznawania twarzy zlecone za pomocą aplikacji webowej,
\item Program zarządzający domem- przekazuje cyklicznie odczytywane dane z czujników oraz wykryte ruchy do bazy danych, w celu dalszej obróbki przez pozostałe moduły.
\end{itemize}
Na usługi pomocnicze wykorzystane w projekcie składają się
\begin{itemize}
\item baza danych- przechowywanie danych o dodanych zadaniach, wynikach, nauczonych sieciach neuronowych oraz osobach,
\item dropbox- przechowywanie większych plików- obrazów oraz nauczonych modeli sieci.
\end{itemize}

\subsection{Wzorzec projektowy MVC}
\begin{figure}[H]
	\centering
	\includegraphics[scale=0.8]{mvc.png}
	\caption{Schemat klasycznego wzorca MVC}
	\label{fig:schemat_mvc}
\end{figure}
Strona internetowa powstała na bazie bardzo popularnego wśród programistów wzorca projektowego MVC. Założenia wzorca Model-Widok-Kontroler są bardzo proste, ich składowymi są:
\begin{itemize}
\item Model- reprezentuje logikę biznesową. Tutaj znajdują się wszelkie obiekty, które służą do wykonywania zaimplementowanej funkcjonalności danej aplikacji,
\item Widok- jest warstwą prezentacji. Odpowiada za prezentację logiki biznesowej (Modelu) użytkownikowi w przystępny sposób,
\item Kontroler- obsługuje żądania użytkownika. Odebrane zadania oddelegowuje do odpowiednich modeli.
\end{itemize}

\subsection{Single Page Application}
Single Page Application (SPA) to aplikacja lub strona internetowa, która w całości wczytuje się za jednym razem. Cały potrzebny do działania strony kod (HTML, CSS, JavaScript) przesyłany jest na początku lub dodawany dynamicznie w kawałkach, zwykle w odpowiedzi na interakcje generowane przez użytkownika.
Sposób działania takiej aplikacji jest zbliżony do odczuć towarzyszących korzystaniu z aplikacji desktopowej lub mobilnej.

\section{Raspberry Pi}
Raspberry Pi jest platformą komputerową stworzoną przez Raspberry Pi Foundation, na którą składa się pojedynczy obwód drukowany. Pierwsza wersja tego urządzenia została zaprezentowana w 2012 roku. Na potrzeby tej pracy wykorzystano nowszą wersję urządzanie w wersji 3 B, którą została wyposażona w 4 rdzeniowy procesor i 1 GB pamięci RAM. Raspberry pozwala na podłączenie wielu urządzeń peryferyjnych za pomocą 4 portów USB lub 40 pinów GPIO. Dodatkowe złącze pozwala na podłączenie dedykowanej kamery. Dzięki znacznej ilości portów GPIO istnieje możliwość komunikacji cyfrowej z sensorami. Budowa urządzenia nie pozwala na podłączenie czujników analogowych.
\begin{figure}[H]
	\centering
	\includegraphics[scale=0.7]{raspberrypi.jpg}
	\caption{Raspberry Pi 3 B}
	\label{fig:raspberrypi}
\end{figure}

\section{Open Cv}
OpenCv (Open Source Computer Vision Library) jest open sourcową biblioteką napisaną w języku C. Udostępniono liczne interfejsy biblioteki pozwalające na pracę z nią miedzy innymi w języku C++ i Python. Biblioteka wspiera systemy operacyjne Linux oraz Windows. Biblioteka została ukierunkowana na przetwarzanie obrazu w czasie rzeczywistym. W licznie udostępnionych funkcjach można znaleźć moduły pozwalające na detekcję i rozpoznawanie twarzy na obrazie, które zostały szerzej opisane w kolejnym punkcie.

\subsection{Detekcja twarzy}
\subsubsection{Haar} \label{haar}
\subsubsection{Deep Neural Network} \label{dnn}

\subsection{Rozpoznawanie twarzy}
\subsubsection{Eigen} \label{eigen}
\subsubsection{Fisher} \label{fisher}
\subsubsection{LBPH} \label{lbph}

\section{Azure}
\subsection{Cognitive Services} \label{cognitive_services}

\section{AWS}



%%%%%%%%%%%%%%%%%%%%%%%%	Przegląd dostępnych metod rozpoznawania twarzy
\chapter{Przegląd dostępnych metod rozpoznawania twarzy}
Przez ostatnie lata dziedzina cyfrowego przetwarzania obrazu bardzo prężnie rozwijała się. W wyniku tego aktualnie dostępnych jest wiele usług pozwalających na uzyskiwanie danych z obrazu. W przypadku tej pracy najbardziej użytecznymi informacjami jest lokalizacja oraz identyfikacja twarzy.
Na rysunku \ref{tab:systemy} przedstawiono cechy kilku wybranych systemów, stworzonych przez jedne z największych firm zajmujących się technologiami informatycznymi na świecie.

\begin{table}[H]\label{tab:systemy}
	\centering
	\caption{Dostępne systemy przetwarzania obrazu}
	\scalebox{0.75}{
	\begin{tabular}{|c|c|c|c|c|c|c|}
  		\hline 
  		 & \bfseries OpenCv & \bfseries Azure CS & \bfseries AWS Rekognition & \bfseries Google & \bfseries face recognition & \bfseries open face\\
  		\hline
  		\bfseries Detekcja twarzy &tak&tak&tak&tak&tak&tak\\
  		\hline
  		\bfseries Identyfikacja twarzy &tak&tak&tak&nie&tak&tak\\
  		\hline
  		\bfseries Rozpoznawanie emocji &nie&tak&nie&nie&nie&nie\\
  		\hline
  		\bfseries SDK &tak&tak&tak&tak&Python&Python\\
  		\hline
  		\bfseries API &nie&tak&tak&tak&nie&nie\\
  		\hline
  		\bfseries licencja &open source&płatna/demo&płatna/demo&płatna/demo&open source&open source\\
  		\hline
  	\end{tabular}
  	}
\end{table}
Zgodnie ze wzrostem popularności usług chmurowych, aktualnie dostępnych jest wiele usług rozpoznawania twarzy, które początkowo były dostępne za pomocą API(Application Programming Interface). Z czasem większość z nich udostępniła również SDK(Software Development Kit) dla najbardziej popularnych języków (między innymi C\#, Java, Python). OpenCv \cite{opencv_doc} jest podstawową open sourcową bibliotekę pozwalającą na identyfikację twarzy. Większość darmowych rozwiązań dostępnych jest na platformie GitHub:
\begin{itemize}
\item face\_ recognition \cite{face_reco_github},
\item open face \cite{open_face}.
\end{itemize}
Niestety żadna z nich nie mogła równać się dojrzałością podobną do OpenCv, a na dodatek większość z nich
można wykorzystać jedynie tworząc oprogramowanie w języku Python.

Poza usługami przedstawionymi w tabeli \ref{tab:systemy}, równie ciekawym rozwiązaniem, aczkolwiek znacznie bardziej rozbudowanym i czasochłonnym jest wykorzystanie Tensorflow \cite{tensorflow} (framework do machine learningu) w celu zbudowanie własnej implementacji sieci neuronowej rozpoznającej twarze.
Podstawowym frameworkiem, o którego użyciu zadecydowano zostało OpenCv. Za wyborem tej biblioteki przemawiała licencja open source, dojrzałość frameworku oraz liczne źródła wiedzy o implementacji. Podstawowe dostępne funkcje i zastosowania opisano w rozdziale \ref{s:open_cv}.

W celu możliwości porównania rozwiązań lokalnych i chmurowych za drugi obiekt badań obrano usługę Azure Cognitive Services \cite{azure}, którą szerzej opisano w rozdziale \ref{azurecs}. O wyborze tego rozwiązania zadecydowała rozbudowana dokumentacja usługi oraz licencja studencka umożliwiająca darmowe wykorzystanie usługi przy zwiększonych limitach zapytań.

\section{Open Cv} \label{s:open_cv}
OpenCv (Open Source Computer Vision Library) jest open sourcową biblioteką napisaną w języku C. Udostępniono liczne interfejsy biblioteki pozwalające na pracę z nią miedzy innymi w języku C++ i Python. Biblioteka wspiera systemy operacyjne Linux oraz Windows. Biblioteka została ukierunkowana na przetwarzanie obrazu w czasie rzeczywistym. W licznie udostępnionych funkcjach można znaleźć moduły pozwalające na detekcję i rozpoznawanie twarzy na obrazie, które zostały szerzej opisane w kolejnym punkcie.

\subsection{Rozpoznawanie twarzy \textcolor{red}{ZRODŁA WZORY I CYTOWANIA}}
Przed zidentyfikowaniem tożsamości, należy najpierw rozwiązać problem detekcji twarzy. Podstawowym sposobem detekcji twarzy wykorzystywanym przez OpenCv jest kaskadowy klasyfikator Haar'a oparty o cechy Haar'a opisane w rozdziale \ref{haar}. Do innych dostępnych rozwiązań wykorzystanych do stworzenia aplikacji należy wytrenowany model głębokiej sieci neuronowej przygotowany przez autorów biblioteki.

\subsubsection{Eigenfaces} \label{eigen}
Eigenfaces nazywany jest również algorytmem twarzy własnych. Głównym postanowieniem algorytmu jest to że nie wszystkie wymiary w zbiorze danych są równie ważne, więc należy znaleźć elementy, w których występuje największa zmienność bo to one mają największą wartość informacyjną. Obraz o wymiarze $p$x$q$ tworzy wektor o wymiarze $m=pq$, więc przykładowo dla wymiarów 100x100 uzyskuje się wektor o wymiarze 10 000. W celu ekstrakcji głównych składowych wykorzystano analizę głównych składowych (ang principal component analysis,PCA), która potrafi przekształcić wielowymiarowy zbiór danych skorelowanych w mniejszy zbiór danych nieskorelowanych. Informację o sposobie działania algorytmu zaczerpnięto z dokumentacji \cite{opencv_doc}.

Działanie algorytmu:
\begin{enumerate}
\item Wyznaczenie wartości średniej $\mu$. Weźmy $n$ wektorów $x_{i}$ reprezentujących obrazy przedstawiające twarz.
\begin{equation}
\mu=\frac{1}{n}\sum_{i=1}^{n}x_{i}
\end{equation}
\item Obliczenie wartości macierzy kowariancji $S$
\begin{equation}
S=\frac{1}{n}\sum^{n}_{i=1}(x_{i}-\mu)(x_{i}-\mu)^{T}
\end{equation}
\item Oblicz wartości eigena $\lambda_{i}$ i wektory eigena $\nu_{i}$ z \begin{equation}
S\nu_{i}=\lambda_{i}\nu_{i},i=1,2,...,n
\end{equation}
\item Posortuj wektory eigena według ich wartości eigena. Główne składowe $k$ są wektorami eigena odpowiadającymi $k$ największym wartościom eigena.
\item Oblicz macierz zawierającą $k$ głównych składowych obserwowanych wektorów $x$.
\begin{equation}
y=W^{T}(x-\mu)
\end{equation}
gdzie $W=(\mu_{1},\mu_{2},...,\mu_{l})$
\end{enumerate}

Identyfikacja twarzy algorytmem Eigenfaces polega na wykonaniu kroków:
\begin{enumerate}
\item Wyznaczenie głównych składowych dla wszystkich obrazów uczących.
\item Wartość głównych składowych zostaje wyznaczona dla badanego zdjęcia.
\item Dla obrazu wejściowego zostaje wyznaczony wektor najbliższy do wartości obliczonych podczas trenowania..
\end{enumerate}

\subsubsection{Fisherfaces} \label{fisher}


Działanie algorytmu:
\begin{enumerate}
\item Weźmy wektor $X$ złożony z próbek należących do $c$ klas
\begin{equation}
X=\{X_{1},X_{1},...,X_{c}\}
X_{i}=\{x_{1},x_{1},...,x_{c}\}
\end{equation}
\item Oblicz macierz rozproszenia wewnątrzklasowego $S_{B}$ i macierz średniego rozproszenia wewnątrzklasowego $S_{W}$
\begin{equation}
S_{B}=\sum_{i=1}^{n}N_{i}(\mu_{i}-\mu)(\mu_{i}-\mu)^{T}
\end{equation}
\begin{equation}
S_{W}=\sum_{i=1}^{n}\sum_{x_{j}\in X_{i}}(x_{j}-\mu_{i})(x_{j}-\mu_{i})^{T}
\end{equation}
gdzie $\mu$ oznacza całkowitą średnią:
\begin{equation}
\mu=\frac{1}{N}\sum_{i=1}^{N}x_{i}
\end{equation}
a $\mu_{i}$ średnią klasy $i\in \{1,...c\}$
\begin{equation}
\mu_{i}=\frac{1}{|X_{i}|}\sum_{x_{j}\in X_{i}}x_{j}
\end{equation}
\item Znajdź wektory bazowe $V$ gdzie $S_{W}$ jest zminimalizowane a $S_{B}$ zmaksymalizowane, gdzie $V$ jest macierzą, której kolumny $\nu_{i}$ są wektorami podstawowymi definiującymi podprzestrzeń
\begin{equation}
J(V)=\frac{\vert V^{T}S_{B}V\vert }{\vert V^{T}S_{W}V\vert }
\end{equation}
\end{enumerate}



Algorytm Fisherfaces opiera się na liniowej analizie dyskryminacyjnej (LDA) i polega na wyznaczeniu wektora cech pozwalającego na rozdzielenie obiektów przynależnych do różnych klas. W przypadku problemu rozpoznawania twarzy, przez klasy obiektów rozumiane są zbiory obrazów przedstawiające tego samego użytkownika. Problem sprowadza się do wyznaczenia wektora, który stanowi przybliżoną granicę między dwiema klasami obiektów, jednak można go uogólnić do problemu wieloklasowego.
Wyznaczanie poszukiwanego wektora rozpoczyna się od wyznaczenia wartości średniej $\nu$ wszystkich obiektów znajdujących się w rozpatrywanym zbiorze, oraz wartości średniej $\nu_{i}$ wewnątrz poszczególnych klas. 
$$
\mu=\frac{1}{N}\sum_{i=1}^{N}x_{i}
$$
Gdzie $N$ oznacza liczebność rozpatrywanego zbioru $x_{i}$
$$
\mu_{i}=\frac{1}{|X_{i}|}\sum_{x_{j}eX_{i}}x_{j}
$$
Gdzie $X_{i}$ jest klasą obiektów o indeksie $i$, dla $i=1,2,...,n$ dla $n$ oznaczającego liczbę klas. Następnie wyznaczana jest macierz rozproszenia wewnątrzklasowego $S_{B}$ oraz macierz średniego rozproszenia wewnątrzklasowego $S_{W}$.
$$
S_{B}=\sum_{i=1}^{n}N_{i}(\mu-\mu_{i})(\mu-\mu_{i})^{T}
$$
$$
S_{W}=\sum_{i=1}^{n}\sum_{x_{j}eX_{i}}(x_{j}-\mu_{i})(x_{j}-\mu_{i})^{T}
$$
Rozwiązanie problemu sprowadza się do znalezienia danych, które pozwolą na otrzymanie największej wartości określającej stosunek rozproszenia wewnątrzklasowego do średniego wewnętrznego rozproszenia klas. Stąd algorytm poszukuje wektora $d$, dla którego poniższa funkcja osiąga maksimum.
$$
f(d)=\frac{d^{T}S_{B}d}{d^{T}S_{W}d}
$$
W celu zmniejszenia złożoności problemu, stosuje się modyfikację powyższego kryterium, wykorzystując macierz $W$ złożona z $k$ głównych wektorów własnych wyznaczonych dla macierzy kowariancji $C$ wszystkich elementów zbioru.
$$
f(d)=\frac{d^{T}W^{T}S_{B}d}{d^{T}S_{W}d}
$$
Znalezienie wektora $d$ pozwala na wyznaczenie optymalnego kierunku rozdzielającego dwie klasy obiektów.

\subsubsection{Local Binary Patterns Histograms} \label{lbph}
W odróżnieniu od holistycznego podejścia w dwóch poprzednich metodach, algorytm histogramów lokalnych binarnych wzorców wykorzystuje lokalne cechy przetwarzanych obiektów.
Dla każdego piksela wyznaczany jest ciąg binarny na podstawie porównania wartości z każdym z sąsiadów. W przypadku, gdy jego wartośc jest większa wtedy przyjmuje wartość 1, a 0 w przypadku przeciwnym. Stąd dla każdego piksela wyznaczana jest wartość $p$-znakowego binarnego ciągu, nazywanego lokalnym binarnym wzorcem.  Wyznaczanie można przeprowadzić w otoczeniu o dowolnym promieniu. W ogólności wartość funkcji LBP dla piksela o współrzędnych $(x_{c},y_{c})$ wyznacza się następująco: 
$$
LBP(x_{c},y_{c})=\sum_{p=0}^{p-1}2^{p}S(i_{p}-i_{c})
$$
Gdzie $p$ jest rozmiarem rozpatrywanego sąsiedztwa o środku w punkcie $(x_{c},y_{c})$ i jasności o wartości $i_{c}$, a $i_{n}$ jest wartością jasności dla $n$- tego punktu sąsiedztwa.
$S(x)$ jest funkcją znaku definiowaną następująco:
$$
S(x) = \left\{ \begin{array}{ll}
1 & \textrm{ dla $x>0$}\\
0 & \textrm{ w p.p.}
\end{array} \right.
$$
Działanie algorytmu polega na podzieleniu obrazu wejściowego na $m$ różnych części i wyznaczenia dla niego wartości LBP jasności pikseli. Następnie dla każdego regionu wyznaczany jest histogram wyliczonych wartości. Tak wyznaczone wartości są konkatenowane do postaci wektora.
Dla próbki wejściowej zostaje wyznaczony wektor histogramów, który następnie jest porównywany z wektorami obrazów użytych do uczenia sieci. Przewidywana etykieta jest wyznaczana na podstawie etykiety wektora najbliższego sąsiada.

\section{Azure Cognitive Services}\label{azurecs}
Cognitive Services jest częścią platformy Azure stworzonej przez Microsoft. Usługa jest płatna, ale w przypadku użytku na potrzeby studenckie przyznawany jest darmowy dostęp na ograniczony czas. Cognitive Services zajmuje się rozwiązywaniem problemów biznesowych dzięki sztucznej inteligencji. Do dostępnych modułów między innymi należą:
\begin{itemize}
\item obraz- algorytmy przetwarzania obrazów umożliwiające inteligentne identyfikowanie, podpisywanie i moderowanie grafik,
\item mowa- konwertowanie wypowiedzi audio na tekst, weryfikacja głosowa,
\item język- przetwarzanie języka naturalnego.
\end{itemize}
Na cele tego projektu wykorzystano moduł dotyczący przetwarzania obrazu, a dokładniej funkcjonalność wykrywania i rozpoznawania twarzy. Producent zadbał o możliwość integracji z większością popularnych języków programistycznych poprzez udostępnienie paczek deweloperskich. W przypadku braku SDK(Software Development Kit) dla wybranego języka istnieje możliwość skorzystania z udostępnionego REST Api. Na stronie producenta można znaleźć obszerną dokumentację \cite{acs_doc} oraz tutoriale.

\subsection{Detekcja twarzy}
Detekcja twarzy opiera się o pojedyncze zapytane do Api Azure'a. Dobierając odpowiednie parametry wejściowe klient może uzyskać dodatkowe informacje związane z przesłanym zdjęciem. W podstawowym przypadku odpowiedź serwisu ogranicza się do przydzielenia identyfikatora dla twarzy oraz obszaru zawierającego twarz w postaci JSON'a.

\subsection{Rozpoznawanie twarzy}
Na potrzeby rozpoznawania twarzy stworzono funkcja LargeGroup, która odpowiada za zarządzanie grupami ludzi/profili. Użytkownik może dodać dowolną ilość osób do grupy, a następnie przypisać wybrane zdjęcia do tożsamości.Na podstawie utworzonej grupy zostaje stworzony model umożliwiający rozpoznawania tożsamości osób w niej zawartej. Sposób wykorzystania tej usługi został znacznie szerzej opisany w rozdziale \ref{trenowanie_azure}. Sposób działania usługi bazuje nauczeniu maszynowym.

\section{AWS Rekognition}
Usługa Rekognition jest odpowiednikiem Azure Cognitive Services ograniczonym do rozwiązań związanych z przetwarzaniem obrazu i video. Do wielu dostępnych funkcji należy analiza obrazu, detekcja twarzy i tekstu, porównywanie oraz rozpoznawanie twarzy. Dla wybranych języków programistycznych udostępniono SDK oraz obszerną dokumentację z licznymi przykładami kodu.


%%%%%%%%%%%%%%%%%%%%%%%%	Budowa systemu
\chapter{System zarządzania metodami rozpoznawania twarzy}
\section{Budowa aplikacji}
Aplikację, którą utworzono na potrzeby tej pracy można podzielić na 3 moduły przedstawione na rysunku \ref{fig:schemat_systemu}. Użyte moduły to:
\begin{itemize}
\item Aplikacja webowa- interfejs pozwalający na zlecanie nowych zadań aplikacji konsolowej i odczyt wyników przesłanych przez obie aplikacje konsolowe,
\item Aplikacja konsolowa- przetwarza zadania detekcji oraz rozpoznawania twarzy zlecone za pomocą aplikacji webowej, trenuje algorytmy wybranym zestawem danych,
\item Aplikacja konsolowa zarządzająca domem- przekazuje cyklicznie odczytywane dane z czujników i kamery.
\end{itemize}
\begin{figure}[H]
	\centering
	\includegraphics[scale=0.7]{schemat_systemu.png}
	\caption{Budowa systemu zarządzania domem}
	\label{fig:schemat_systemu}
\end{figure}
We wczesnej fazie projektu istniała jedna aplikacja konsolowa ale ze względu na złożoność programu przetwarzającego zadania została ona rozdzielona na aplikację, która musi być uruchamiana na Raspberry Pi (program zarządzający domem) oraz drugą, którą można uruchomić w dowolnym innym środowisku. Wykorzystane usługi pomocnicze to:
\begin{itemize}
\item baza danych MSSql- przechowywanie danych o zadaniach, wynikach, nauczonych sieciach oraz profilach,
\item Dropbox- przechowywanie obrazów oraz nauczonych modeli sieci,
\item Azure Cognitive Services- usługa przetwarzająca obrazy (rozdział \ref{azurecs}).
\end{itemize}

\section{Konfiguracje uruchomieniowe}
W celu zmaksymalizowania wydajności modułów, podjęto decyzję o uruchomieniu każdego z nich w odrębnym środowisku. System został uruchomiony w dwóch konfiguracjach. Niezależnie dla konfiguracji niezmienny pozostał moduł aplikacji konsolowej zarządzającej domem, który ze względu na wymaganie fizycznego dostępu do urządzeń podłączonych do Raspberry Pi zawsze był uruchamiany w tym środowisku.
Pierwsza z konfiguracji została oparta o środowisku chmurowe Azure, a druga AWS. Usługi dedykowane dla poszczególnych modułów przedstawiono na rysunku \ref{fig:schemat_azure} i \ref{fig:schemat_aws}.

\begin{figure}[H]
	\centering
	\includegraphics[scale=0.5]{budowa_azure.png}
	\caption{Podział systemu na usługi Azure}
	\label{fig:schemat_azure}
\end{figure}

\begin{figure}[H]
	\centering
	\includegraphics[scale=0.5]{budowa_aws.png}
	\caption{Podział systemu na usługi AWS}
	\label{fig:schemat_aws}
\end{figure}



%%%%%%%%%%%%%%%%%%%%%%%%	Aplikacja internetowa
\chapter{Aplikacja internetowa}
Aplikacja webowa jest jedyną częścią systemu, do której użytkownik może mieć bezpośredni dostęp. Strona powstała w celu maksymalnego uproszczenia procesu badania kolejnych algorytmów i usług, które różniły się sposobem podawania danych wejściowych, sposobem uczenia oraz formatem zwracanych odpowiedzi. Do pozostałych zalet takiego rozwiązania należy ułatwienie przechowywania danych, poprzez umieszczenie ich we wspólnym miejscu co pomaga, w późniejszej interpretacji wyników.
\begin{figure}[H]
	\centering
	\includegraphics[scale=0.7]{aplikacja_webowa_razor_widok.png}
	\caption{Wygląd strony głównej aplikacji internetowej (rozwiązanie 1)}
	\label{fig:strona_glowna_razor}
\end{figure}
\pagebreak
Zgodnie z interfejsem przedstawionym na rysunku \ref{fig:strona_glowna_razor}, strona została podzielona na 5 głównych sekcji:
\begin{itemize}
\item Notifications- powiadomienia,
\item Detections- detekcje,
\item Profiles- profile tożsamości,
\item Face Recognizers- algorytmy rozpoznawania twarzy,
\item Recognitions- identyfikacja tożsamości,
\item Readings- odczyty sensorów.
\end{itemize}

\section{Technologie}\label{s:web_technologie}
Aplikacja webowa powstała w najnowszej kompilacji .NET Core 2, będącej międzyplatformową strukturą open source o wysokiej wydajności służącą do tworzenia nowoczesnych aplikacji internetowych opartych na usługach chmurowych. Logika biznesowa aplikacji została zaprogramowana w języku C\#. Warstwa widoku powstała w dwóch dostępnych rozwiązaniach, nieznacznie różniących się wyglądem, ale znacznie odbiegających od siebie sposobem działania. Przed omówieniem poszczególnych rozwiązań przedstawiono, krótkie definicje wykorzystanego wzorca projektowego MVC oraz technologii SPA. 

\subsection{Rozwiązanie 1- Razor Pages}
Pierwsza wersja została oparta o strony tworzone w technologi Razor Pages opartej o składnię Razor oraz podstawowe technologie webowe: HTML i CSS. Taki sposób tworzenia warstwy prezentacji jest zalecany dla aplikacji .NET Core, ponieważ pozwala zminimalizować ilość pracy wymaganej na jej utworzenie oraz zapewnia bardzo prosty proces wdrożenia. Aplikacja utworzona z pomocą Razor'a została zaprezentowana na rysunku \ref{fig:strona_glowna_razor}.

\subsection{Rozwiązanie 2- Angular 4}
Druga wersja widoku aplikacji oferuje dostęp do tych samych możliwości co pierwsze rozwiązanie, ale powstała przy pomocy frameworka webowego- Angular 4. Strona główna widoczna jest na rysunku \ref{fig:strona_glowna_angular}. Angular jest open sourcowym frameworkiem używanym do tworzenia aplikacji SPA (Single Page Application), napisany w języku TypeScript i wspierany oraz rozwijany przez Google.
\begin{figure}[H]
	\centering
	\includegraphics[scale=0.6]{aplikacja_webowa_angular_widok.png}
	\caption{Wygląd strony głównej aplikacji internetowej (rozwiązanie 2)}
	\label{fig:strona_glowna_angular}
\end{figure}

\section{Detekcja twarzy}
Detections jest stroną odpowiedzialną za wykrywanie twarzy na obrazach przesłanych do systemu. Na głównej stronie możemy zobaczyć wszystkie zlecone detekcje, zarówno nowe jak i już zakończone.
\begin{figure}[H]
	\centering
	\includegraphics[scale=0.6]{detections.png}
	\caption{Widok detekcji twarzy}
	\label{fig:detections}
\end{figure}
Przycisk 'New' widoczny na \ref{fig:detections} pozwala na stworzenie nowego zadania wykrycia twarzy na obrazie, które zostanie przetworzone przez moduł aplikacji konsolowej. Na formularzu z rysunku \ref{fig:new_detection} należy podać nazwę zadania oraz za pomocą przycisku 'Wybierz plik' wybrać obraz w formacie png,jpg lub jpeg znajdujący się na dysku użytkownika.
\begin{figure}[H]
	\centering
	\includegraphics[scale=0.6]{new_detection.png}
	\caption{Tworzenie nowej detekcji}
	\label{fig:new_detection}
\end{figure}
Użycie przycisku 'Show' widocznego przy każdym zadaniu detekcji na rysunku \ref{fig:detections}, pozwoli na wyświetlenie szczegółów związanych z requestem, w tym wyników jeśli zadanie zostało zakończone.
\begin{figure}[H]
	\centering
	\includegraphics[scale=0.5]{detekcja_z_wynikami.png}
	\caption{Widok zakończonej detekcji}
	\label{fig:detekcja_zakonczona}
\end{figure}

\pagebreak

Strona ze zdjęcia \ref{fig:detekcja_zakonczona} pozwala sprawdzić datę utworzenia oraz zakończenia zadania, obraz wejściowy oraz tabelę z informacjami o wynikach detekcji, na które składają się:
\begin{itemize}
\item detection type- typ detekcji (haar, głęboka sieć neuronowa, azure),
\item faces found- ilość wykrytych twarzy,
\item processing time- czas potrzebny na wykrycie twarzy daną metodą,
\item startX, endX, startY, endY- współrzędne lewego dolnego oraz prawego górnego wierzchołka prostokąta obejmującego wykrytą twarz,
\item Area- rozmiar obszaru obejmującego twarz w pikselach.
\end{itemize}

\section{Profile}
Strona 'Profiles' służy do tworzenia nowych profili, które później mogą zostać wykorzystane jako dane uczące podczas trenowania wybranymi algorytmami. Do każdego profilu osoby musi zostać przypisany zasób minimum 2 zdjęć. W przypadku braku możliwości wykrycia twarzy na zdjęciu, zostanie ono zignorowane podczas procesu nauczania.
\begin{figure}[H]
	\centering
	\includegraphics[scale=0.6]{people.png}
	\caption{Lista utworzonych profili}
	\label{fig:people}
\end{figure}
Nowa osoba może zostać utworzona w podobny sposób jak request detekcji twarzy. Jedyną różnicą jest wymóg wyboru kilku obrazów.
\begin{figure}[H]
	\centering
	\includegraphics[scale=0.6]{person.png}
	\caption{Widok profilu}
	\label{fig:person}
\end{figure}
Utworzona osoba nie może być modyfikowana. Pierwsze załadowanie widoku osoby może trwać wydłużony czas z powodu procesu generowania linków do plików magazynowanych w usłudze Dropbox.

\section{Algorytmy rozpoznawania twarzy}
W zakładce 'Face Recognizers' użytkownik może utworzyć zadanie trenowania modelu wybranym algorytmem (OpenCv: Fisherfaces, Eigenfaces, LBPH, Azure: Cognitive Services) używając wybranej grupy profili z zakładki 'Profiles'. Użytkownik nie ma możliwości wyboru typu sieci, która zostanie nauczona. Domyślnie wszystkie dostępne rozwiązania (wymienione powyżej) zostaną wykorzystane.
\begin{figure}[H]
	\centering
	\includegraphics[scale=0.6]{neural_networks.png}
	\caption{Strona przedstawiająca istniejące grupy usług rozpoznawania tożsamości}
	\label{fig:sieci_neuronowe}
\end{figure}
Podczas tworzenia nowej sieci każdy istniejący profil zostanie wyświetlony jako checkbox, który należy zaznaczyć jeśli dany profil ma zostać wykorzystany w procesie nauczania. W wyniku szybkiego rozrostu bazy profili dodano checkbox 'Check all' pozwalający na wybranie wszystkich profili poprzez jedno kliknięcie. Kolejnym parametrem, który musi zostać wypełniony jest 'Photos per person used' określający maksymalną ilość zdjęć przypisanych do danego profilu, które mogą zostać wykorzystane do nauki sieci. Podana wartość powinna znajdować się w przedziale od 2 do 20. W przypadku gdy niedostępna będzie określana ilość zdjęć w danym profilu, to wykorzystane zostanie tyle zdjęć ile jest dostępnych.
\begin{figure}[H]
	\centering
	\includegraphics[scale=0.5]{nowa_siec.png}
	\caption{Tworzenie nowej grupy algorytmów identyfikacji twarzy}
	\label{fig:nowa_siec}
\end{figure}
Wykorzystując przycisk 'Show' z widoku na rysunku \ref{fig:sieci_neuronowe} można wyświetlić szczegółowe informacje o wybranej grupie sieci neuronowych. Dostępny widok ukazano na rysunku \ref{fig:siec_neuronowa}. Widoczne są na nim wszystkie parametry grupy,na które składają się: ilość wykorzystanych zdjęć i profili oraz tabela z informacjami o przygotowanych sieciach. Informacje dostępne w tabeli to nazwa i typ sieci, rozmiar utworzonego modelu, czas potrzebny na wytrenowanie oraz pełny czas(czas trenowania + przygotowanie danych wejściowych) potrzebny na stworzenie sieci.
\begin{figure}[H]
	\centering
	\includegraphics[scale=0.5]{siec_neuronowa.png}
	\caption{Szczegółowe informacje o utworzonej grupie}
	\label{fig:siec_neuronowa}
\end{figure}

\section{Rozpoznawanie tożsamości}
W sekcji 'Recognitions' użytkownik ma możliwość wykorzystać wcześniej utworzone zbiory identyfikatorów w celu rozpoznania tożsamości na zdjęciu przedstawiającym pojedynczą osobę.
\begin{figure}[H]
	\centering
	\includegraphics[scale=0.6]{recognitions.png}
	\caption{Lista zadań identyfikacji osoby}
	\label{fig:recognitions}
\end{figure}
Podobnie jak na pozostałych stronach, podczas tworzenia nowego zadania użytkownik będzie musiał uzupełnić prosty formularz. W formularzu przedstawionym na rysunku \ref{fig:new_recognition} należy załączyć jedno zdjęcie oraz wybrać grupę algorytmów trenowaną tymi samymi danymi, która ma zostać wykorzystana do identyfikacji tożsamości osoby.
\begin{figure}[H]
	\centering
	\includegraphics[scale=0.6]{new_recognition.png}
	\caption{Formularz tworzenia zadania identyfikacji}
	\label{fig:new_recognition}
\end{figure}
Po zakończonym procesie identyfikacji opisanym w kolejnym podrozdziale, użytkownik może wyświetlić wynik uzyskany przez każdą sieć dostępną w grupie. Przykładowy rezultat widoczny jest na zdjęciu \ref{fig:recognition}.
\begin{figure}[H]
	\centering
	\includegraphics[scale=0.6]{recognition.png}
	\caption{Zakończony request identyfikacji}
	\label{fig:recognition}
\end{figure}

\section{Odczyty sensorów}
Zakładka Readings powstała w celu czytelnej prezentacji odczytów zebranych z sensorów w dniach funkcjonowania systemu. Odczyty zbierane są co minutę co generuje znaczną ilość wpisów do wyświetlenia. Z tego powodu po wejściu na stronę należy wybrać dzień, którym zainteresowany jest użytkownik. Przykładową listę zaprezentowano na rysunku \ref{fig:all_days}. Po wybraniu dnia ukazuje się widok, na którym przedstawiono wszystkie odczyty zebrane w wybranym okresie czasu.Do dostępnych informacji należą temperatura, wilgotność powietrza oraz czas wykonania pomiaru.
\begin{figure}[H]
	\centering
	\includegraphics[scale=0.7]{all_days.png}
	\caption{Widok pozwalający na wybór dnia}
	\label{fig:all_days}
\end{figure}
\begin{figure}[H]
	\centering
	\includegraphics[scale=0.7]{daily_readings.png}
	\caption{Widok wyświetlający wszystkie odczyty z wybranego dnia}
	\label{fig:daily_readings}
\end{figure}

\section{Powiadomienia} \label{notifications}
Ostatnią ze stworzonych zakładek są Powiadomienia. Widok służy do wyświetlania wszystkich powiadomień utworzonych przez system, na które składają się powiadomienia sensorów oraz wykrycia ruchu.
\begin{figure}[H]
	\centering
	\includegraphics[scale=0.7]{notifications.png}
	\caption{Widok wyświetlający wszystkie odczyty z wybranego dnia}
	\label{fig:notifications}
\end{figure}
 Typy oraz sposób ich generowania zostały szerzej opisany w rozdziale \ref{aplikacja_zarzadzanie}. Każde wyświetlane powiadomienie posiada wiadomość,typ oraz czas utworzenia. W przypadku wykrycia ruchu dodatkowo wyświetlone zostaje zdjęcie uchwyconego wydarzenia. Kilka przykładowych powiadomień można zaobserwować na rysunku \ref{fig:notifications}.
 


%%%%%%%%%%%%%%%%%%%%%%%%	Aplikacja konsolowa
1\chapter{Aplikacja konsolowa}
Aplikacja konsolowa powstała w celu przetwarzania czasochłonnych zadań w tle, tak by użytkownik aplikacji webowej nie doświadczał długich czasów ładowania oraz ewentualnych błędów podczas przerwania sesji. Aplikacja konsolowa uruchamiana jest co 5 minut i przetwarza zadania trzech typów w kolejności widocznej na rysunku \ref{fig:worker_proces}. Poszczególne procesy zostały szerzej opisane w kolejnych podrozdziałach.
\begin{figure}[H]
	\centering
	\includegraphics[scale=0.6]{worker_proces.png}
	\caption{Proces działania aplikacji konsolowej}
	\label{fig:worker_proces}
\end{figure}

\section{Technologie}
Aplikacja konsolowa powstała w języku programowania Python w wersji trzeciej. W początkowej fazie projektu za wyborem tego języka przemawiała wieloplatformowość, możliwości wprowadzania szybkich zmian w kodzie i brak potrzeby go kompilowania. C\#, który został wybrany do stworzenia aplikacji webowej okazał się nie przystosowany do modyfikowania kodu na platformie Raspberry Pi z powodu braku dostępnego .NET Core SDK na procesory ARM, a uruchomienie programu wymagało znacznie większej ilości zasobów obliczeniowych niż Python. Ostatecznie aplikacja konsolowa została przeniesiona na zewnętrzny serwer. Popularność Pythona pozwoliła na
integrację z wybranymi usługami dzięki dostępności SDK (Software Developmnet Kit).

\section{Proces wykrywania twarzy}
Uogólniony proces detekcji twarzy na obrazie został przedstawiony na grafie \ref{fig:wykrywanie_proces}.
Proces przetwarzania zadań detekcji rozpoczyna się od pobrania z bazy wszystkich żądań o statusie 'New'. Następnie każde zadanie przetwarzane jest osobno. Dla aktualnie procesowanego zadania pobierany jest obraz wejściowy z usługi Dropbox i zapisywany w lokalnym folderze. Następnym krokiem jest wywołanie procesu odpowiedzialnego za preprocessing zdjęcia i pozyskanie oczekiwanych wyników detekcji. Krok ten został szerzej opisany w podrozdziałach \ref{detekcja_haar}, \ref{detekcja_dnn} i \ref{detekcja_azure}. Po uzyskaniu wyników każdą dostępną w programie metodą, utworzony zostaje obraz wyjściowy dla każdej metody (haar, azure, ..), który program uploaduje do odpowiedniego folderu w usłudze Dropbox. Po poprawnym wgraniu plików wynikowych następuje zapisanie informacji o rezultatach w bazie danych. Po wykonaniu zadania bez żadnych błędów request zostaje oznaczony jako zakończony. W innym przypadku status zostaje zmieniony na 'Error'. Proces powtarzany jest dla każdego wpisu pozyskanego z bazy.
\begin{figure}[H]
	\centering
	\includegraphics[scale=0.6]{wykrywanie_twarzy.png}
	\caption{Proces wykrywania twarzy zaimplementowany w aplikacji konsolowej}
	\label{fig:wykrywanie_proces}
\end{figure}

\subsection{OpenCv Haar} \label{detekcja_haar}
Pierwszą z metod detekcji twarzy, która została zintegrowana z programem jest detekcja metodą Haar'a. Algorytm został opisany w rozdziale \ref{haar}. Przed uruchomieniem detekcji obraz wejściowy należy odpowiednio przygotować. W tym celu wcześniej pobrany obraz zostaje wczytany do programu i przekonwertowany do odcieni szarości. Tak przygotowany obraz można poddać detekcji. Detektor zwraca współrzędne obszarów zawierających w sobie twarz. Tak uzyskane dane należy przekonwertować do formatu, który został przyjęty jako wspólny dla wszystkich metod.
\begin{figure}[H]
	\centering
	\includegraphics[scale=0.6]{detekcja_haar.png}
	\caption{Wykrywanie twarzy metodą Haar zaimplementowane w aplikacji konsolowej}
	\label{fig:wykrywanie_haar}
\end{figure}

\subsection{OpenCv DNN (Deep Neural Network)} \label{detekcja_dnn}
Proces detekcji z wykorzystaniem głęboko uczonej sieci neuronowej nie wymaga formatowania obrazu do skali szarości, ale za to obraz musi zostać przeskalowany do odpowiedniego, wcześniej przyjętego rozmiaru. Dodatkową zaletą tej metody jest format odpowiedzi detektora, który oprócz obszaru zawierającego twarz zwraca również 'pewność' z jaką twarz została wykryta, co pozwala na odfiltrowanie wyników poniżej pewnego poziomu.
\begin{figure}[H]
	\centering
	\includegraphics[scale=0.6]{detekcja_dnn.png}
	\caption{Wykrywanie twarzy metodą DNN (Deep Neural Network) zaimplementowane w aplikacji konsolowej}
	\label{fig:wykrywanie_dnn}
\end{figure}

\subsection{Azure Cognitive Services} \label{detekcja_azure}
Podczas detekcji twarzy przy pomocy Azure Cognitive Services wszystkie operacje przetwarzania obrazu wykonywane są w chmurze co pozwala odciążyć środowisko rozruchowe. Proces wykrycia twarzy ogranicza się do przesłania wybranego obrazu do ACS Api (Application programming interface). Lokalnie nie jest wykonywany żaden preprocessing. W odpowiedzi klient Api uzyskuje wiadomość JSON zawierającą skonfigurowane podczas zapytania parametry. JSON zostaje zparsowany do postaci obiektu ze wszystkimi informacjami. Wartości zostają przekonwertowane do formatu [startX, startY, endX, endY].
\begin{figure}[H]
	\centering
	\includegraphics[scale=0.6]{detekcja_azure.png}
	\caption{Wykrywanie twarzy używając Azure Cognitive Services zaimplementowane w aplikacji konsolowej}
	\label{fig:wykrywanie_azure}
\end{figure}

\section{Proces trenowania modelu dla wybranych algorytmów \textcolor{red}{cos o tej sieci napisac lub omowic jej zrodla}}
Uproszczony proces trenowania grupy identyfikatorów twarzy przedstawiono na grafie \ref{fig:trenowanie_proces}. Po jego zakończeniu użytkownik otrzymuję grupę z wytrenowanymi modelami gotową do wykorzystania w zakładce 'Recognitions'.
Proces zaczyna się od sprawdzenia jakie dane uczące (profile utworzone w zakładce 'Profiles' aplikacji webowej) są dostępne lokalnie. Repozytorium zostaje porównane z danymi dostępny w bazie, a następnie zaktualizowane o brakujące profile.
\begin{figure}[H]
	\centering
	\includegraphics[scale=0.6]{proces_nauczania.png}
	\caption{Proces trenowania sieci zaimplementowany w aplikacji konsolowej}
	\label{fig:trenowanie_proces}
\end{figure}
W kolejnym kroku następuje pobranie z bazy wszystkich requestów, które nie zostały jeszcze zakończone. Następnie dla każdego zgłoszenia z listy wykonywany jest proces przygotowania listy danych uczących, które wybrano podczas tworzenia zadania w aplikacji webowej. Kolejne dwa kroki są specyficzne dla danego algorytmu dlatego 'trenuj OpenCv FaceRecognizers' opisano w rozdziale \ref{trenowanie_open_cv}, a 'trenuj AzureLargeGroup' w \ref{trenowanie_azure}. Po zakończeniu trenowania zintegrowanych algorytmów następuje proces zapisu danych o utworzonych modelach sieci w bazie danych.
W przypadku braku komplikacji, request zostaje oznaczony jako zakończony. Proces zostaje powtórzony dla każdego zadania wczytanego do listy na początku grafu \ref{fig:trenowanie_proces}.

\subsection{Trenowanie identyfikatorów OpenCv} \label{trenowanie_open_cv}
Ogólny schemat procesu trenowania identyfikatorów twarzy OpenCv (Eigenfaces, Fisherfaces, LHGP) został przedstawiony na rysunku \ref{fig:trenowanie_open_cv}. Zalecany proces trenowania udostępniony w dokumentacji \cite{opencv_doc} został poddany zmianą wymaganym przez aplikację konsolową i zaimplementowany.

Proces rozpoczyna się od stworzenia listy zawierającej wszystkie przypisane do sieci profile oraz załączone do nich obrazy. Następnie dla każdego zdjęcia z listy zostaje wykonany preprocessing. Pierwszym krokiem jest próba wykrycia twarzy na obrazie. W przypadku nie znalezienia twarzy, obraz zostaje zignorowany i program przechodzi do przetwarzania kolejnego obrazu. Jeśli na zdjęciu zostanie zlokalizowana twarz, to obszar ją zawierający zostaje przeskalowany do odcieni szarości, a następnie przekonwertowany do postaci numpy array. Wektor zawierający informację o twarzy zostaje dodany do listy uczącej wraz z odpowiadającym mu ID profilu, do którego należy zdjęcie. Po przetworzeniu wszystkich obrazów na liście, zostaje utworzone i nauczone wszystkie wcześniej wymienione identyfikatory dostępne w OpenCv.
\begin{figure}[H]
	\centering
	\includegraphics[scale=0.6]{trenowanie_open_cv.png}
	\caption{Trenowanie algorytmów z biblioteki OpenCv zaimplementowane w aplikacji konsolowej}
	\label{fig:trenowanie_open_cv}
\end{figure}

\subsection{Trenowanie identyfikatora Azure} \label{trenowanie_azure}
Proces rozpoczyna się od wczytania wszystkich osób oraz ich zdjęć, które zostały przypisane do zadania trenowania sieci. Na diagramie \ref{fig:trenowanie_azure} niebieskim kolorem oznaczono zapytania do Azure Cognitive Services API(Application Programming Interface).

Zgodnie z dokumentacją \cite{acs_doc} pierwszym i najważniejszym krokiem jest utworzenie nowej AzureLargeGroup, do której w kolejnym kroku zostaną dodane id/nazwy wszystkich profili znajdujących się na wczytanej liście.  Następnie każdy obraz z listy zostaje przypisany odpowiedniej osobie w AzureLargeGroup. W tym rozwiązaniu klient nie musi martwić się detekcją twarzy na obrazie, bo jest ona wykonywana przez usługę Azure po przesłaniu obrazu. W przypadku problemów z wykryciem twarzy na obrazie, zostanie zwrócona informacja mówiąca o tym że plik zostanie zignorowany podczas procesu nauczania. Po dodaniu wszystkich zdjęć, program wywołuje funkcję trenowania nowej sieci na podstawie danych dołączonych do AzureLargeGroup. Już po około sekundzie, identyfikator tożsamości udostępniony przez usługę Azure jest gotowy do działania.

Ze względu na setki zapytań, które muszą zostać wykonane do zewnętrznego serwisu Azure, proces przygotowania danych przed trenowaniem może wydłużyć się nawet do kilku minut w przypadku wybrania dużej ilości profili i zdjęć.
\begin{figure}[H]
	\centering
	\includegraphics[scale=0.55]{trenowanie_azure.png}
	\caption{Trenowanie usługi Azure Cognitive Services zaimplementowane w aplikacji konsolowej}
	\label{fig:trenowanie_azure}
\end{figure}

\section{Proces rozpoznawania twarzy}\label{s:proces_rozpoznawania}
Ogólny proces rozpoznawania twarzy przedstawiony na rysunku \ref{fig:rozpoznawanie_proces} jest zbliżony do wcześniej opisywanych procesów detekcji i trenowania. 

Proces rozpoczyna się od pobrania wszystkich wytrenowanych modeli, których aktualnie nie ma w lokalnym katalogu. W kolejnym etapie zostają wczytane wszystkie nowe zadania rozpoznania twarzy. Dla każdego z requestów na liście wykonywane są te same czynności, a pierwszą z nich jest pobranie pliku wejściowego z usługi Dropbox. Następnie twarz zostaje rozpoznana za pomocą każdego dostępnego algorytmu identyfikacji. 

Poszczególne metody zostały szerzej opisane w kolejnych podrozdziałach. Po ukończeniu identyfikacji tożsamości, wszystkie uzyskane wyniki zostają wprowadzone do bazy danych, a zadanie oznaczone jako ukończone. Gdy zadania na liście skończą się to dochodzi do zakończenia procesu.
\begin{figure}[H]
	\centering
	\includegraphics[scale=0.6]{rozpoznawanie_twarzy.png}
	\caption{Proces identyfikacji tożsamości zaimplementowany w aplikacji konsolowej}
	\label{fig:rozpoznawanie_proces}
\end{figure}

\subsection{Identyfikacja tożsamości metodami dostępnymi w OpenCv}
Niezależnie od typu FaceRecognizera, proces rozpoznawania twarzy z użyciem algorytmów OpenCv wygląda identycznie. Przykładowe zastosowanie zostało opisane w dokumentacji \cite{opencv_doc}. 

Obraz zostaje wczytany do programu, a następnie zostaje podjęta próba detekcji twarzy metodą Haara. W przypadku nie wykrycia żadnej twarzy do zadania zostaje dodany pusty rezultat z komentarzem informującej o braku twarzy na przekazanym obrazie. 
Jeśli na zdjęciu zostanie wykryta twarz to proces przechodzi do kolejnego kroku, którym jest wczytanie modelu odpowiadającego typowi identyfikatora i numerowi wybranej grupy sieci. Przed identyfikacją wycinek obrazu zawierający twarz zostaje przekonwertowany do numpy array, który jest formatem wymaganym przez FaceRecognizer. W rezultacie uzyskana zostaje tożsamość osoby ze zdjęcia prezentowana w postaci numeru Id z bazy oraz wartość wektora oddalenia od najbliższego obiektu w modelu (0.0 oznacza idealne dopasowanie, im większa wartość tym mniejsza pewność). 
\begin{figure}[H]
	\centering
	\includegraphics[scale=0.6]{rozpoznawanie_open_cv.png}
	\caption{Proces identyfikacji osoby wykorzystując identyfikatory OpenCv zaimplementowany w aplikacji konsolowej}
	\label{fig:rozpoznawanie_open_cv}
\end{figure}

\subsection{Identyfikacja tożsamości przy użyciu Azure Cognitive Services}
Proces rozpoznawania twarzy przy pomocy Azure Cognitive Services jest mniej obciążający dla platformy, na której uruchomiono program, ale za to może być wolniejszy od pozostałych rozwiązań ze względu na ograniczenia jakości połączenia.

Kroki procesu identyfikacji zaimplementowano zgodnie z przykładowym procesem przedstawionym w dokumentacji \cite{acs_doc}, a pierwszym z nich jest wczytanie słownika tożsamości przypisanych do AzureLargeGroup. W kolejnym etapie zdjęcie wejściowe zostaje wysłane do usługi detekcji opisanej w \ref{detekcja_azure} ale ze zmodyfikowanymi parametrami. Każdej wykrytej twarzy zostaje nadany numer identyfikacyjny Azure. 

Jeśli jakaś twarz została znaleziona to kolejnym etapem jest wysłanie żądania identyfikacji uzyskanej twarzy przy pomocy AzureLargeGroup, która została przypisana do tego requesta. Tożsamość rozpoznana przez CS zostaje zwrócona w postaci numera identyfikacyjnego usługi, dlatego musi ona zostać porównana z wcześniej wczytanym słownikiem, który pozwala na odczytanie Id osoby istniejącej w bazie danych programu.
\begin{figure}[H]
	\centering
	\includegraphics[scale=0.6]{rozpoznawanie_azure.png}
	\caption{Proces identyfikacji osoby wykorzystując Azure Cognitive Servces zaimplementowany w aplikacji konsolowej}
	\label{fig:rozpoznawanie_azure}
\end{figure}


%%%%%%%%%%%%%%%%%%%%%%%%	Aplikacja konsolowa
\chapter{Aplikacja konsolowa do zarządzania domem} \label{aplikacja_zarzadzanie}
Aplikacja konsolowa do zarządzania domem służy jako aplikacja odpowiedzialna za wszystkie działania związane z Raspberry Pi oraz jego peryferiami. Na te zadania składają się 2 czynności:
\begin{itemize}
\item odczyt wartości sensorów,
\item detekcja ruchu.
\end{itemize}
Obie czynności wykonywane są niezależnie od siebie. Funkcja wykrywania ruchu działa bez przerwy, w przypadku problemu zostanie ponownie uruchomiona po około minucie za pomocą zadania wpisanego do Crontab'a. Odczyt parametrów wskazywanych przez czujniki odbywa się cyklicznie co 1 minutę.
\section{Odczyt wartości sensorów}

\section{Detekcja ruchu}

\subsection{Proces wykrywania ruchu}

%%%%%%%%%%%%%%%%%%%%%%%%	Ocena/Badanie
\chapter{Porównanie wykorzystanych technologii \textcolor{red}{TODO}}
\section{Detekcja twarzy}
W rozdziale tym opisano testy przeprowadzone w celu oceniania skuteczności działania wybranych detektorów twarzy. Podjęto decyzję o wykonaniu kilku zdjęć przedstawiających twarz w różnych pozycjach, częściowo zasłoniętą oraz w zmiennym oświetleniu. Wizualizację wyników przeprowadzonych testów zawarto w tabeli \ref{tab:porownanie_detektorow}.
\begin{longtable}{|c|c|c|c|c|c|} 
\hline
  		& \bfseries Wejście & \bfseries Haar & \bfseries Dnn & \bfseries Azure \\
  		\hline
  		1&		\begin{minipage}{.2\textwidth}
      	\includegraphics[width=\linewidth, height=20mm]{detekcja/3_input.jpg}
    	\end{minipage}
		& 
		\begin{minipage}{.2\textwidth}
      	\includegraphics[width=\linewidth, height=20mm]{detekcja/3_haar.jpg}
    	\end{minipage}
		& 
		\begin{minipage}{.2\textwidth}
      	\includegraphics[width=\linewidth, height=20mm]{detekcja/3_dnn.jpg}
    	\end{minipage}
		& 
		\begin{minipage}{.2\textwidth}
      	\includegraphics[width=\linewidth, height=20mm]{detekcja/3_azure.jpg}
    	\end{minipage}	
		\\
  		\hline
  		2&		\begin{minipage}{.2\textwidth}
      	\includegraphics[width=\linewidth, height=20mm]{detekcja/4_input.jpg}
    	\end{minipage}
		& 
		\begin{minipage}{.2\textwidth}
      	\includegraphics[width=\linewidth, height=20mm]{detekcja/4_haar.jpg}
    	\end{minipage}
		& 
		\begin{minipage}{.2\textwidth}
      	\includegraphics[width=\linewidth, height=20mm]{detekcja/4_dnn.jpg}
    	\end{minipage}
		& 
		\begin{minipage}{.2\textwidth}
      	\includegraphics[width=\linewidth, height=20mm]{detekcja/4_azure.jpg}
    	\end{minipage}	
		\\
  		\hline
  		3& 		\begin{minipage}{.2\textwidth}
      	\includegraphics[width=\linewidth, height=20mm]{detekcja/5_input.jpg}
    	\end{minipage}
		& 
		\begin{minipage}{.2\textwidth}
      	\includegraphics[width=\linewidth, height=20mm]{detekcja/5_haar.jpg}
    	\end{minipage}
		& 
		\begin{minipage}{.2\textwidth}
      	\includegraphics[width=\linewidth, height=20mm]{detekcja/5_dnn.jpg}
    	\end{minipage}
		& 
		\begin{minipage}{.2\textwidth}
      	\includegraphics[width=\linewidth, height=20mm]{detekcja/5_azure.jpg}
    	\end{minipage}	
		\\
  		\hline
  		4&  		\begin{minipage}{.2\textwidth}
      	\includegraphics[width=\linewidth, height=20mm]{detekcja/6_input.jpg}
    	\end{minipage}
		& 
		\begin{minipage}{.2\textwidth}
      	\includegraphics[width=\linewidth, height=20mm]{detekcja/6_haar.jpg}
    	\end{minipage}
		& 
		\begin{minipage}{.2\textwidth}
      	\includegraphics[width=\linewidth, height=20mm]{detekcja/6_dnn.jpg}
    	\end{minipage}
		& 
		\begin{minipage}{.2\textwidth}
      	\includegraphics[width=\linewidth, height=20mm]{detekcja/6_azure.jpg}
    	\end{minipage}	
		\\
  		\hline
  		5&  		\begin{minipage}{.2\textwidth}
      	\includegraphics[width=\linewidth, height=20mm]{detekcja/7_input.jpg}
    	\end{minipage}
		& 
		\begin{minipage}{.2\textwidth}
      	\includegraphics[width=\linewidth, height=20mm]{detekcja/7_haar.jpg}
    	\end{minipage}
		& 
		\begin{minipage}{.2\textwidth}
      	\includegraphics[width=\linewidth, height=20mm]{detekcja/7_dnn.jpg}
    	\end{minipage}
		& 
		\begin{minipage}{.2\textwidth}
      	\includegraphics[width=\linewidth, height=20mm]{detekcja/7_azure.jpg}
    	\end{minipage}	
		\\
  		\hline
  		6&  		\begin{minipage}{.2\textwidth}
      	\includegraphics[width=\linewidth, height=20mm]{detekcja/8_input.jpg}
    	\end{minipage}
		& 
		\begin{minipage}{.2\textwidth}
      	\includegraphics[width=\linewidth, height=20mm]{detekcja/8_haar.jpg}
    	\end{minipage}
		& 
		\begin{minipage}{.2\textwidth}
      	\includegraphics[width=\linewidth, height=20mm]{detekcja/8_dnn.jpg}
    	\end{minipage}
		& 
		\begin{minipage}{.2\textwidth}
      	\includegraphics[width=\linewidth, height=20mm]{detekcja/8_azure.jpg}
    	\end{minipage}	
		\\
  		\hline
  		7&  		\begin{minipage}{.2\textwidth}
      	\includegraphics[width=\linewidth, height=20mm]{detekcja/9_input.jpg}
    	\end{minipage}
		& 
		\begin{minipage}{.2\textwidth}
      	\includegraphics[width=\linewidth, height=20mm]{detekcja/9_haar.jpg}
    	\end{minipage}
		& 
		\begin{minipage}{.2\textwidth}
      	\includegraphics[width=\linewidth, height=20mm]{detekcja/9_dnn.jpg}
    	\end{minipage}
		& 
		\begin{minipage}{.2\textwidth}
      	\includegraphics[width=\linewidth, height=20mm]{detekcja/9_azure.jpg}
    	\end{minipage}	
		\\
  		\hline
  		8&  		  		\begin{minipage}{.2\textwidth}
      	\includegraphics[width=\linewidth, height=20mm]{detekcja/11_input.jpg}
    	\end{minipage}
		& 
		\begin{minipage}{.2\textwidth}
      	\includegraphics[width=\linewidth, height=20mm]{detekcja/11_haar.jpg}
    	\end{minipage}
		& 
		\begin{minipage}{.2\textwidth}
      	\includegraphics[width=\linewidth, height=20mm]{detekcja/11_dnn.jpg}
    	\end{minipage}
		& 
		\begin{minipage}{.2\textwidth}
      	\includegraphics[width=\linewidth, height=20mm]{detekcja/11_azure.jpg}
    	\end{minipage}	
		\\
  		\hline
  		9&  		  		\begin{minipage}{.2\textwidth}
      	\includegraphics[width=\linewidth, height=20mm]{detekcja/12_input.jpg}
    	\end{minipage}
		& 
		\begin{minipage}{.2\textwidth}
      	\includegraphics[width=\linewidth, height=20mm]{detekcja/12_haar.jpg}
    	\end{minipage}
		& 
		\begin{minipage}{.2\textwidth}
      	\includegraphics[width=\linewidth, height=20mm]{detekcja/12_dnn.jpg}
    	\end{minipage}
		& 
		\begin{minipage}{.2\textwidth}
      	\includegraphics[width=\linewidth, height=20mm]{detekcja/12_azure.jpg}
    	\end{minipage}	
    	\\
  		\hline 
\caption{Porównanie działania detektorów w wybranych warunkach}
\label{tab:porownanie_detektorow}
\end{longtable}
Pierwszą różnicą, którą można zaobserwować w sposobie działania detektorów jest rozmiar obszaru twarzy oznaczany przez algorytmy. Najmniejszym z nich jest twarz wykrywana metodą Haara, oznaczany obszar zaczyna się na wysokości brwi i kończy niewiele poniżej ust. W przypadku detekcji twarzy z wykorzystaniem głębokiej sieci neuronowej prostokąt obejmuje także całe czoło oraz podbródek przedstawionej osoby. Rozmiar obszaru uzyskanego przez ACS(Azure Cognitive Services) mieści się pomiędzy metodą Haara a DNN(Deep Neural Network), obejmuję częściowo czoło i podbródek.

Każda z metod dobrze poradziła sobie w przypadku centralnego zdjęcia twarzy przy dobrym jak i słabym oświetleniu. Najskuteczniejszą z metod okazała się metoda DNN, która wykryła twarz na każdym z dziewięciu zdjęć. Metoda wykorzystująca platformę Azure okazała się nieskuteczna w przypadku zdjęcia przedstawiającego częściowo zakrytą twarz oraz z odchyloną głową. 

Najwrażliwszym na jakoś zdjęcia okazała się metoda Haara, która nie była w stanie sobie poradzić choćby z najmniejszymi zmianami w układzie twarzy na zdjęciu. Używając tego detektora nie udało się wykryć twarzy na obrazach, które nie sprawiły problemu pozostałym metodą. Między innymi nie wykryto twarzy na obrazie przedstawiającym profil oraz głowę przechyloną na bok.

Zbadano, że wysoka skuteczność detekcji twarzy metodą DNN wiąże się z większą ilością fałszywych detekcji w porównaniu z pozostałymi metodami. Z tego powodu wyniki uzyskane tą metodą należy filtrować,a jest to możliwe poprzez ustawienie pewności powyżej, której algorytm ma zwracać odpowiedzi.

Średnie czasy odpowiedzi detektorów przedstawiono w tabeli \ref{tab:systemy}. W najkrótszym czasie uzyskano odpowiedź od sieci neuronowej, a najwolniejsza zgodnie z oczekiwania była metoda wykorzystująca ACS. Znacznie dłuższy czas trwania wynika z konieczności skontaktowania się z serwerem z innego kraju.

\begin{table}[H]\label{tab:systemy}
	\centering
	\caption{Średni czas przetwarzania zadania detekcji twarzy}
	\scalebox{1.0}{
	\begin{tabular}{|c|c|c|c|}
  		\hline 
  		 & \bfseries Haar & \bfseries Dnn & \bfseries ACS\\
  		\hline
  		\bfseries Średni czas odpowiedzi [s]& 0,023127109 &0,018201053 &0,465438843 \\
  		\hline
  	\end{tabular}
  	}
\end{table}

W celu potwierdzenia uzyskanych wniosków przeprowadzono końcowy test polegający na uruchomieniu każdego z detektorów na bazie profili składającej się z 2000 zdjęć. Wyniki przedstawiono na rysunku \ref{fig:porownanie_detektorow}. Detektor oparty o głęboką siec neuronową wykrył twarz na 94,6\% zdjęć, Azure 90,1\%, a Haar 88,7\%. Wyniki potwierdziły, że sieć neuronowa, w tym przypadku daje najlepsze rezultaty, korzystniejsze o nawet 6\% od pozostałych metod.

\begin{figure}[H]
	\centering
	\includegraphics[scale=1.0]{porownanie_detektorow.png}
	\caption{Porównanie działania detektorów na bazie 2000 zdjęć}
	\label{fig:porownanie_detektorow}
\end{figure}

\section{Dane treningowe}
Za dane wejściowe do procesu trenowania sieci neuronowych wybrano \fnurl{Glasgow Unfamiliar Face Database (GUFD)}{http://www.facevar.com/glasgow-unfamiliar-face-database}, która dostępna jest za darmo i można jej używać na potrzeby badań uczelnianych oraz publikacji. Jedynym warunkiem użycia jest zacytowanie jednej z publikacji właściciela bazy.

Baza zawiera 303 tożsamości. Na każdą z tożsamości składa się 20 zdjęć jednej osoby wykonanych w różnych warunkach np. różniące się kąty ujęcia, wyrazy twarzy oraz z dodatkowymi akcesoriami(okulary, kaptur, czapka). 

Do materiałów uczących dodatkowo dodano profil autora tej pracy magisterskiej.

\section{Trenowanie sieci neuronowych}
Podczas badania sieci neuronowych postanowiono sprawdzić kilka podstawowych czynników, na które składa się wpływ ilości wybranych tożsamości oraz wpływ ilości zdjęć przydzielonych tożsamości na:
\begin{itemize}
\item czas potrzebny na preprocessing danych uczących,
\item czas trwania trenowania modelu,
\item rozmiar pliku zawierającego model (jeśli istnieje).
\end{itemize}
W rozdziale \ref{b:rozpoznawanie} omówiono wpływ wyżej wymienionych parametrów na czas oraz pewność identyfikowania tożsamości.

\subsection{Wpływ wybranych parametrów na przygotowanie danych uczących}
Zgodnie z oczekiwaniami, preprocessing danych wymagany przed trenowaniem identyfikatorów powiązanych z biblioteką OpenCv jest znacznie szybszy od Azure. Największą wadą wynikającą z chmurowości  ACS(Azure Cognitive Services) jest konieczność wykonania tysięcy zapytań, w tym większość z nich odpowiadających za przekazanie zdjęć do chmury. Prędkość dodawania danych jest ograniczana przez jakość połączenia z serwerem. Na przygotowanie 19 zdjęć dla 100 profili, system potrzebował około 15 minut, które może się wydawać astronomiczne w porównaniu z 24 sekundami potrzebnymi na przygotowanie danych dla metod trenowanych lokalnie.

Na rysunku \ref{fig:czas_p_profile} można zaobserwować że czas potrzebny na przygotowanie danych w przypadku OpenCv rósł liniowo. Podobny efekt był oczekiwany dla ACS. Brak liniowości może wynikać z różnych rozmiarów plików oraz chwilowej lepszej/gorszej jakości połączenia z serwerem.
\begin{figure}[H]
	\centering
	\includegraphics[scale=1.0]{czas_przygotowania_a_ilosc_profili.png}
	\caption{Wpływ ilości profili użytych podczas treningu na czas przygotowania danych uczących}
	\label{fig:czas_p_profile}
\end{figure}
Wyniki uzyskane podczas drugiego testu, które przedstawiono na grafie \ref{fig:czas_p_zdjecia} potwierdzają tezę o niestabilności połączenia z serwisem chmurowym. Dla tego testu udało się uzyskać zależność ilości zdjęć od czasu zbliżoną do liniowej.
\begin{figure}[H]
	\centering
	\includegraphics[scale=1.0]{czas_przygotowania_a_ilosc_zdjec.png}
	\caption{Wpływ ilości zdjęć przydzielonych profilowi na czas przygotowania danych uczących}
	\label{fig:czas_p_zdjecia}
\end{figure}

\subsection{Wpływ wybranych parametrów na czas trwania treningu}
Kolejnym interesującym testem, który przeprowadzona było zbadanie wpływu ilości użytych profili oraz ilości zdjęć przypisanych profilowi na czas trwania nauki. Zgodnie z oczekiwaniami wraz z przyrostem danych zwiększał się czas wymagany na wytrenowanie sieci. Wyniki przedstawiono na rysunku \ref{fig:czas_t_profile}. Wyjątkiem pozostało rozwiązanie chmurowe Azure, które niezależnie od ilości danych zawsze trwało sekundę (najmniejsza jednostka czasu, którą można uzyskać od usługi). Podobnym zachowaniem wykazał się algorytm LBPH (\ref{lbph}), w przypadku którego czas treningu wydłuż się od 0,7s do maksymalnie 12s. W przypadku pozostałych metod czas treningu wzrósł z początkowej wartości maksymalnie kilku sekund do nawet kilku minut
\begin{figure}[H]
	\centering
	\includegraphics[scale=1.0]{czas_trenowania_a_ilosc_profili.png}
	\caption{Wpływ ilości profili użytych podczas treningu na czas trenowania sieci}
	\label{fig:czas_t_profile}
\end{figure}
Bliźniaczy test przeprowadzony dla zmiennej ilości zdjęć przypisanej do profilu zakończył się wynikami zbliżonymi do poprzedniego badania. Przyrost czasu pozostał nieliniowy.
\begin{figure}[H]
	\centering
	\includegraphics[scale=1.0]{czas_trenowania_a_ilosc_zdjec.png}
	\caption{Wpływ ilości zdjęć przydzielonych profilowi na czas trenowania sieci}
	\label{fig:czas_t_zdjecia}
\end{figure}

\subsection{Wpływ wybranych parametrów na rozmiar modelu sieci}
Wyniki uzyskane podczas badaniu wpływu parametrów na rozmiar utworzonego modelu przedstawiono na rysunku \ref{fig:rozmiar_profile} oraz \ref{fig:rozmiar_zdjecia}. Ze względu na brak dostępu do informacji o rozmiarze modelu, Azure Cognitve Services został pominięty podczas tego testu.

W badaniu z poprzedniego rozdziału okazało się, że proces trenowania algorytmu Eigen jest najbardziej czasochłonny. Ta informacja ma swoje odzwierciedlenie w rozmiarze modelu. Rozmiar modelu uzyskany dla maksymalnej ilości wykorzystanych profili osiągnął wartość aż 3GB, które w porównaniu do 163MB dla LBPH oraz 197MB dla metody Fishera jest ogromną wartością. W uproszczeniu można by powiedzieć że wzrost rozmiaru pliku był liniowy i zwiększał się dwukrotnie wraz z dwukrotnym zwiększeniem się ilości wykorzystanych profili. 
\begin{figure}[H]
	\centering
	\includegraphics[scale=1.0]{rozmiar_modelu_a_ilosc_profili.png}
	\caption{Wpływ ilości profili użytych podczas treningu na rozmiar utworzonego modelu}
	\label{fig:rozmiar_profile}
\end{figure}
W przypadku badania wpływu ilości zdjęć przypisanych do profilu na rozmiar pliku, wystąpił problem z uzyskaniem pierwszej próbki dla metody Fishera, która jak się okazało wymaga minimalnie 3 zdjęć przypisanych do profilu. Podobnie jak w poprzednim teście algorytm Eigen oraz LBPH zachowywał się liniowo co jest dobrze widoczne na rysunku \ref{fig:rozmiar_zdjecia}. Interesującym zachowaniem wykazał się model utworzony dla metody Fishera, który już przy 5 zdjęciach osiągnął wartość minimalnie różniącą się od maksymalnej. Dodawanie kolejnych zdjęć każdemu z profili powodowało minimalne zmiany w rozmiarze pliku. Uzyskane zachowanie znacznie różniło się od wyników z testu poprzedniego (rysunek \ref{fig:rozmiar_profile}).
\begin{figure}[H]
	\centering
	\includegraphics[scale=1.0]{rozmiar_modelu_a_ilosc_zdjec.png}
	\caption{Wpływ ilości zdjęć przydzielonych profilowi na rozmiar utworzonego modelu}
	\label{fig:rozmiar_zdjecia}
\end{figure}

\section{Rozpoznawanie twarzy} \label{b:rozpoznawanie}
Kolejnym etapem badań było przetestowanie działania identyfikatorów twarzy, Testy podzielono na 3 kategorie:
\begin{itemize}
\item wpływ ilości zdjęc przypisanych do profilu na pewność rozpoznania,
\item wpływ ilości profili użytych podczas treningu na pewność rozpoznania,
\item wpływ atrybutów oraz błędna identyfikacja,
\item test wszystkich próbek
\end{itemize}
\subsection{Wpływ ilości zdjęć przypisanych do profilu na pewność rozpoznania}
Metody dostępne w bibliotece OpenCv różnią się od Azure sposobem określania pewności identyfikacji. ACS (Azure Cognitive Services) zwraca pewność w postaci procentowej, a OpenCv jako wektor oddalenia od najbliższej próbki, którego im wartość jest mniejsza tym lepiej.

W pierwszym etapie wykorzystano sieci nauczone podczas badań z rozdziału poprzedniego do identyfikacji jednego ze zdjęć. Wybrane zdjęcie zostało poprawnie zidentyfikowane przez każdą z metod.

Dla Azure, którego wyniki przedstawiono na rysunku \ref{fig:azure_zdjecia}. Dla wybranego zdjęcia jedynie 2 próbki dla każdego ze 100 profili zapewniły wysoki poziom pewności równy około 93\%. Największe zmiany w pewności identyfikacji zaszły przy zmianie ilości próbek od 2 do 10 na profil. Po przekroczeniu tej wartości pewność ustabilizowała się w okolicy 96%. 
\begin{figure}[H]
	\centering
	\includegraphics[scale=1.0]{azure_pewnosc_a_ilosc_zdjec.png}
	\caption{Wpływ ilości zdjęć przydzielonych profilowi na pewność identyfikacji Azure. Użyto 100 profili}
	\label{fig:azure_zdjecia}
\end{figure}
Podobną sytuację zaobserwowano dla sieci z biblioteki OpenCv przedstawionych na rysunku \ref{fig:opencv_zdjecia}. Wraz ze wzrostem ilości zdjęć malała wartość wektora odległości, którego wartość ustaliła się na 0 po przekroczeniu 10 zdjęć dla każdej z metod. Niewyjaśnionym zjawiskiem pozostaje chwilowy wzrost niepewności dla metody Eigen między ilością zdjęć 2, a 3. Najmniej pewną metodą okazał się algorytm Eigen, którego niepewność w szczytowych momentach była nawet 1,5 raza większa od algorytmu Fishera. Niezależnie od ilości zdjęć, dla testowanego obrazu metoda LBHP była pewna lub wartość wektora oddalenia był zbliżona do zera.
\begin{figure}[H]
	\centering
	\includegraphics[scale=1.0]{opencv_pewnosc_a_ilosc_zdjec.png}
	\caption{Wpływ ilości zdjęć przydzielonych profilowi na pewność identyfikacji OpenCv. Użyto 100 profili}
	\label{fig:opencv_zdjecia}
\end{figure}

\subsection{Wpływ ilości profili użytych podczas treningu na pewność rozpoznania}
Z powodu zbyt dużej pewności identyfikacji każdej z metod dla ilości zdjęć przypisanych profilowi powyżej 10, podczas tego badania wykorzystano sieci podczas treningu, których wykorzystano maksymalnie 5 zdjęć na profil. 

Początkowe wyniki uzyskane dla Azure były na tyle zaskakujące, że wykonano test dla dwóch próbek. Dane przedstawiono na rysunku \ref{fig:azure_5__profile}. Azure Cognitive Services wykazało, że jest odporne na poszerzanie zbioru o kolejne profile. Zmiana ilości profili z pięciu na sto nie wywołała żadnej zmiany w pewności identyfikacji.
\begin{figure}[H]
	\centering
	\includegraphics[scale=1.0]{5_azure_pewnosc_a_ilosc_profili.png}
	\caption{Wpływ ilości profili użytych podczas nauki na pewność identyfikacji Azure. Użyto 5 zdjęć dla każdego profilu}
	\label{fig:azure_5__profile}
\end{figure}
Większość metod udostępnionych przez OpenCv nie wykazała się cechami zbliżonymi do Azure. Wraz ze wzrostem ilości użytych profili, niepewność identyfikacji wzrastała. Wyniki są widoczne na rysunku \ref{fig:opencv_5__profile}. Największy wzrost zaobserwowano dla metody Eigen. W przypadku algorymu Fishera widoczny jest znaczny skok niepewności podczas zmiany ilości profili z 5 na 10 ale przy kolejnych zmianach nie dochodziło do równie gwałtownych zmian. Test metody LBPH wykazał zerową wartość wektora odległości dla każdej ilości profili.
\begin{figure}[H]
	\centering
	\includegraphics[scale=1.0]{5_opencv_pewnosc_a_ilosc_profili.png}
	\caption{Wpływ ilości profili użytych podczas nauki na pewność identyfikacji OpenCv. Użyto 5 zdjęć dla każdego profilu}
	\label{fig:opencv_5__profile}
\end{figure}

\subsection{Wpływ dodatkowych atrybutów oraz błędna identyfikacja}
Na potrzeby testu wszystkie sieci zostały nauczone profilami, którym przypisano 5 zdjęć. Podczas treningu dołączono dodatkowy profil zawierający osobę z zarostem przedstawioną w tabeli \ref{tab:porownanie_detektorow}. W celu utrudnienia poprawnej identyfikacji sieci zostały przetestowane obrazem bez zarostu widocznym na rysunku \ref{fig:bledna_identyfikacja}.
\begin{figure}[H]
	\centering
	\includegraphics[scale=0.5]{bledna_identyfikacja.png}
	\caption{Wyniki uzyskane podczas testu dodatkowego atrybutu w postaci zarostu}
	\label{fig:bledna_identyfikacja}
\end{figure}
Jedyną metodą, która prawidłowo rozpoznała twarz bez zarostu jest Azure. Interesującą informacją, którą można uzyskać z tego testu jest potwierdzenie różnic w sposobie działania każdej z metod identyfikacji. Każdy algorytm OpenCv rozpoznał zdjęcie jako całkiem inną tożsamość. Jest to pierwszy przypadek,w której uzyskano niepewność większą od zera dla metody LBPH. Pozostałe wartości niepewności znacznie się zwiększyły względem wartości wektorów odległości, które uzyskano w poprzednich testach podczas poprawnej identyfikacji.

Znacznie korzystniej rozwiązano problem błędnej identyfikacji w usłudze Azure. Podczas problemów z identyfikacją serwis zwróci informację o tym, że twarz nie została rozpoznana.

\subsection{Wpływ ilości zdjęć przypisanych do profilu na skuteczność działania}
W tym rozdziale opisano końcowy test polegający na przetestowaniu 100 profili po 20 zdjęć każdy sieciami wytrenowanymi odpowiednio 5,10,19 zdjęciami z zasobu dwudziestu.

Wyniki uzyskane podczas testu detektorów (patrz rysunek \ref{fig:porownanie_detektorow}) pokazały, że ilość twarzy wykrywanych metodą Haara jest znacznie mniejsza od pozostałych metod, co mogłoby przynieść niezadowalające wyniki dla metod oparty o bibliotekę OpenCv, w wyniku których wyciągnięto by nieprawidłowe wnioski. W związku z tym problemem podjęto decyzję o wytrenowaniu dodatkowych sieci neuronowych z biblioteki OpenCv, które do detekcji twarzy podczas treningu oraz identyfikacji wykorzystują głęboką sieć neuronową.

Wszystkie przeprowadzone testy umieszczono na wykresie \ref{fig:porownanie_identyfikator} w celu łatwiejszej interpretacji danych.

Przeprowadzony test usługi ACS (Azure Cognitive Services) potwierdził, że w tym przypadku do poprawnej identyfikacji większości obrazów wystarczający jest zasób danych trenujących ograniczony do tylko 5 zdjęć na profil. Warto zauważyć, że dla pierwszego testu różnica poprawnych identyfikacji między ACS, a najlepszą siecią neuronową z biblioteki OpenCv wyniosła około 300. W przypadku najgorszej z nich różnica była bliska 600.

Zgodnie z przypuszczeniami sieci neuronowe OpenCv wykorzystujące Dnn osiągnęły lepsze wyniki od tych korzystających z detektora Haara. Jedynie algorytm Fisher(Dnn) przy 5 próbkach nie rozpoznał więcej twarzy niż wszystkie metody wykorzystujące drugi sposób jej detekcji.

Zwiększenie ilości próbek do 10 przyniosło wzrost skuteczności identyfikacji każdej sieci. Dla tego przypadku testowego, każda sieć wykorzystujące detektor Dnn była lepsza od używających detektora Haara. Największy przyrost skuteczności osiągnięto dla sieci neuronowych wywodzących się z biblioteki OpenCv w połączeniu z detekcją wykorzystującą głęboką sieć neuronową, ale żadna z nich nie przekroczyła poziomu osiągniętego przez ACS

Test przeprowadzony dla 19 próbek przypisanych do każdego ze stu profili wywołał  wzrost skuteczności identyfikacji w większym lub mniejszym stopniu dla każdej metody. Zgodnie z oczekiwaniami wykorzystanie Dnn do detekcji twarzy pozwoliło na osiągnięcie lepszych wyników końcowych niż Azure Cognitive Services.
\begin{figure}[H]
	\centering
	\includegraphics[scale=1.0]{porownanie_identyfikacji.png}
	\caption{Porównanie działania usług rozpoznających na bazie 2000 zdjęć}
	\label{fig:porownanie_identyfikator}
\end{figure}
Dane z wykresu \ref{fig:porownanie_identyfikator} dla próbki o rozmiarze 5 i 19 zdjęć przedstawiono w postaci procentowej w tabeli \ref{tab:skutecznosc_identyfikacji}.
Największa skuteczność identyfikacji osiągnęła sieć neuronowa typu Lbph(Dnn) i Eigen(Dnn). Największy wzrost skuteczności wraz ze wzrostem ilości próbek zanotowano dla metody Fisher(Dnn). Ilość próbek miała najmniejszy wpływ na skuteczność identyfikacji opartej o Azure Cognitive Services. Najniższą skutecznością działania wykazał się algorytm Fisher(Haar).

\begin{table}[H]\label{tab:skutecznosc_identyfikacji}
	\centering
	\caption{Skuteczność identyfikacji wybranych metod}
	\scalebox{1.0}{
	\begin{tabular}{|c|c|c|c|}
  		\hline 
  		 & \bfseries 5 & \bfseries 19\\
  		\hline
  		\bfseries Azure Cognitive Services& 88,7\% & 90,2\% \\
  		\hline
  		\bfseries Fisher (Haar)& 63,65\% & 76,5\% \\
  		\hline
  		\bfseries Fisher (Dnn)& 63,85\% & 93,5\% \\
  		\hline
  		\bfseries Eigen (Haar)& 63,95\% & 78,65\% \\
  		\hline
  		\bfseries Eigen (Dnn)& 72,9\% & 94,35\% \\
  		\hline
  		\bfseries Lbph (Haar)& 67,5\% & 78,65\% \\
  		\hline
  		\bfseries Lbph (Dnn)& 75,5 & 94,35 \\
  		\hline
  	\end{tabular}
  	}
\end{table}

\subsection{Czas przetwarzania zapytania}
Dane dotyczące czasu wymaganego na uzyskanie odpowiedzi od każdej z z metod identyfikacji umieszczono w tabeli \ref{tab:szybkosc_identyfikacji}. Ponownie Azure okazał się najwolniejszy, prawdopodobnie z powodu potrzeby łączenia się z zewnętrznym serwerem. Najszybszą odpowiedź można uzyskać wykorzystując metodę Eigen.

Porównując dane z tabeli z rysunkiem \ref{fig:rozmiar_zdjecia} można zauważyć że prędkość odpowiedzi powiązana jest z rozmiarem utworzonego modelu. Metoda posiadająca największy model odpowiada najszybciej, a najmniejszy najwolniej.
\begin{table}[H]\label{tab:szybkosc_identyfikacji}
	\centering
	\caption{Średni czas przetwarzania zadania identyfikacji twarzy}
	\scalebox{1.0}{
	\begin{tabular}{|c|c|}
	  	\hline 
	  	 &\bfseries Średni czas odpowiedzi [s]\\
  		\hline 
	  	\bfseries Lbph&1,07648058255514\\
  		\hline 
	  	\bfseries Eigen&0,0814295546213786\\
  		\hline 
	  	\bfseries Fisher&0,576785226662954\\
  		\hline 
	  	\bfseries Azure Cognitive Services&1,1062401757724\\
  		\hline 
  	\end{tabular}
  	}
\end{table}

\subsection{Przydatność w zastosowaniu dla IoT}


\section{Ocena przydatności wybranych usług IoT}



Konfiguracja bazy danych oraz maszyny wirtualnej okazała się równie prosta w każdym ze środowisk. Proces konfiguracji odbywał się poprzez wypełnienie kilku formularzy niewymagających wprowadzania dużej ilości informacji (ze względu na ograniczenia studenckiej licencji). Na końcu procesu uzyskany zostaje connection string oraz konto za pomocą, którego można zalogować się na serwer.

Największa różnica między dwoma dostawcami wystąpiła w przypadku usług hostujących aplikację webową czyli Azure App Service oraz AWS Elastic Beanstalk. Podczas pierwszych testów konfiguracji aplikacja internetowa istniała jedynie w rozwiązaniu przygotowanym w języku Angular 4. Struktura aplikacji została przygotowana na podstawie wzoru przygotowanego przez Microsoft. Azure App Service bezproblemowo wspierał nawet najnowsze wersje rozwiązań przygotowanych dla .NET Core 2. Niestety AWS nie był przygotowany 


%%%%%%%%%%%%%%%%%%%%%%%%	Podsumowanie
\chapter{Podsumowanie}
Celem tej pracy było stworzenie ,,Systemu bezpieczeństwa i zarządzania domem w IoT''. Cel pracy zrealizowano poprzez utworzenie systemu aplikacji internetowej oraz dwóch aplikacji konsolowych opartych o wybrane języki programistyczne (C#, Python) i kilka platform: Raspberry Pi oraz rozwiązania chmurowe.
Utworzony system pozwala na kontrolę kilku podstawowych aspektów inteligentnego domu, na które składa się nadzór stanu czujników oraz wykrycie ruchu przez kamerę, która przykładowo mogłaby zostać umieszczona przed wejściem do mieszkania lub domu.
Na potrzeby kontroli wymienionych cech utworzono system powiadomień polegający na rejestracji zdarzeń kluczowych dla systemu, jak np. przekroczenie pewnego poziomu temperatury lub wykrycie ruchu przed domem. W kolejnym etapie system mógłby zostać rozbudowany o powiadomienia e-mail oraz aplikację mobilną pozwalającą na odebranie ważnego powiadomienia w każdym miejscu i chwili.

Podczas tworzenia systemu postanowiono wykorzystać dwóch największych dostawców usług chmurowych: Amazon i Azure. Każdy z nich udostępnił szeroki zakres usług wspierających tworzenie i działanie nowoczesnych aplikacji dla IoT. Podczas przeprowadzonych testów żaden z dostępnych systemów nie wykazał przewagi nad usługami konkurenta. Obie platformy zapewniają szeroki zakres dokumentacji, paczek deweloperskich oraz licencję demo pozwalającą na zapoznanie się z wybranymi usługami w ograniczonym zakresie. Przewagę jednej z platform mogłoby wykazać porównanie pozostałych usług (między innymi IoT Hub oraz AWS IoT) ale z powodu dynamicznego wzrostu popularności systemów IoT, ilość usług wzrasta równie dynamicznie, a ograniczenia czasowe nie pozwoliły na przetestowanie ich wszystkich.

Aspektem systemu bezpieczeństwa, na którym postanowiono się skupić była integracja i porównanie wybranych rozwiązań pozwalających na detekcję oraz rozpoznawanie twarzy. Postanowiono porównać metody dostępne w bibliotece open source- OpenCv oraz płatnej usługi- Azure Cognitive Services. W celu częściowej automatyzacji procesu uczenia oraz testowania wybranych algorytmów przygotowano odpowiedni interfejs internetowy pozwalający między innymi na tworzenie profili i trenowanie sieci neuronowych. W przyszłości system można by zamienić w system kontroli dostępu oparty o identyfikację na podstawie twarzy. Podczas badań porównano prędkość działania, ilość wymaganych zasobów oraz skuteczność trzech algorytmów OpenCv oraz ACS (Azure Cognitive Serices).


%%%%%%%%%%%%%%%%%%%%%%%%		BIBLIOGRAFIA
\addcontentsline{toc}{chapter}{Bibliografia}
%\newpage
\begin{thebibliography}{99}
\bibitem{1}
\emph{Arduino Playground},
http://playground.arduino.cc/
\bibitem{seb} ktos:
\emph{tytul},
Wydawnictwo , Poznań 2000
\end{thebibliography}

\nocite{*}
\printbibliography 


%%%%%%%%%%%%%%%%%%%%%%%%		Dodatki
\chapter*{Dodatki}
\addcontentsline{toc}{chapter}{Dodatki}
Dodatek do niniejszej pracy stanowi płyta CD zawierająca:
\begin{itemize}
\item pracę dyplomową w postaci źródłowej (LaTeX),
\item pracę dyplomową w postaci pliku pdf.
\end{itemize}

\newpage
\addcontentsline{toc}{section}{Spis rysunków}	
\listoffigures

\newpage
\addcontentsline{toc}{section}{Spis tablic}	
\listoftables

%%%%%%%%%%%%%%%%%%%%%%%%

\end{document}