Celem tej pracy było stworzenie ,,Systemu bezpieczeństwa i zarządzania domem w IoT''. Cel pracy zrealizowano poprzez utworzenie systemu aplikacji internetowej oraz dwóch aplikacji konsolowych opartych o wybrane języki programistyczne (C\#, Python) i kilka platform: Raspberry Pi oraz rozwiązania chmurowe.
Utworzony system pozwala na kontrolę kilku podstawowych aspektów inteligentnego domu, na które składa się nadzór stanu czujników oraz wykrycie ruchu przez kamerę, która przykładowo mogłaby zostać umieszczona przed wejściem do mieszkania lub domu.
Na potrzeby kontroli wymienionych cech utworzono system powiadomień polegający na rejestracji zdarzeń kluczowych dla systemu, jak np. przekroczenie pewnego poziomu temperatury lub wykrycie ruchu przed domem. W kolejnym etapie system mógłby zostać rozbudowany o powiadomienia e-mail oraz aplikację mobilną pozwalającą na odebranie ważnego powiadomienia w każdym miejscu i chwili.

Podczas tworzenia systemu postanowiono wykorzystać dwóch największych dostawców usług chmurowych: Amazon i Azure. Każdy z nich udostępnił szeroki zakres usług wspierających tworzenie i działanie nowoczesnych aplikacji dla IoT. Podczas przeprowadzonych testów żaden z dostępnych systemów nie wykazał przewagi nad usługami konkurenta. Obie platformy zapewniają szeroki zakres dokumentacji, paczek deweloperskich oraz licencję demo pozwalającą na zapoznanie się z wybranymi usługami w ograniczonym zakresie. Przewagę jednej z platform mogłoby wykazać porównanie pozostałych usług (między innymi IoT Hub oraz AWS IoT) ale z powodu dynamicznego wzrostu popularności systemów IoT, ilość usług wzrasta równie dynamicznie, a ograniczenia czasowe nie pozwoliły na przetestowanie ich wszystkich.

Aspektem systemu bezpieczeństwa, na którym postanowiono się skupić była integracja i porównanie wybranych rozwiązań pozwalających na detekcję oraz rozpoznawanie twarzy. Postanowiono porównać metody dostępne w bibliotece open source- OpenCv oraz płatnej usługi- Azure Cognitive Services. W celu częściowej automatyzacji procesu uczenia oraz testowania wybranych algorytmów przygotowano odpowiedni interfejs internetowy pozwalający między innymi na tworzenie profili i trenowanie sieci neuronowych. W przyszłości system można by zamienić w system kontroli dostępu oparty o identyfikację tożsamości na podstawie twarzy. Podczas badań porównano prędkość działania, ilość wymaganych zasobów oraz skuteczność trzech algorytmów OpenCv oraz ACS (Azure Cognitive Serices). Po zestawieniu wyników detekcji oraz rozpoznawania twarzy najskuteczniejszym rozwiązaniem okazało się połączenie sieci neuronowej do detekcji twarzy oraz algorytmu Lbph do rozpoznawania. Niewiele niższą skuteczność uzyskano dla usługi ACS. 

Jednak odpowiedź, która z metod identyfikacji jest lepsza nie jest tak oczywista. Podczas implementacji konkretnego systemu, np. wcześniej wymienionego systemu dostępu należałoby wziąć pod uwagę moc obliczeniową platformy oraz rozmiar dysku. Największą zaletą metod z biblioteki OpenCv jest ich licencja, niestety posiadają również wiele wad, a największymi z nich jest długi czas trenowania, wymóg posiadania przestrzeni do przechowywania nauczonych modeli oraz konieczność powtórzenia długiego procesu treningu, który jest uzależniony od mocy obliczeniowej platformy w celu dodania nawet pojedynczego profilu. W rozwiązaniu chmurowym ACS użytkownik nie musi się przejmować wymienionymi problemami. Inicjalizacja systemu może być czasochłonna, ale jest on otwarty na dynamiczną rozbudowę dzięki procesowi nauczania trwającemu około sekundy, a do osiągnięcia skuteczności na zadowalającym poziomie wymagana jest znacznie mniejsza ilość danych uczących niż dla OpenCv. 

Technologia dąży w kierunku rozwiązań chmurowych, jakość połączenia internetowego nie jest już problemem dlatego Azure Cognitive Services i usługi chmurowe jest kierunkiem, w którym powinno się udać IoT.

Jednym z problemów napotkanych podczas powstawania pracy było wybranie języka programowania, który pozwoliłby na zintegrowanie wszystkich wybranych technologii i uruchomienie programu w dowolnym środowisku. Kolejnym problemem była bardzo ograniczona moc obliczeniowa początkowej platformy, która znacznie wydłużała proces trenowania sieci neuronowych. Z tego powodu system podzielono na dwie aplikacje konsolowe uruchamiane w osobnych środowiskach.

Praca ta pozwoliła na zapoznanie się  z aktualnymi rozwiązaniami używanymi w dziedzinie IoT oraz rozpoznawania twarzy. Na kolejnym etapie rozwoju system mógłby zostać zintegrowany z większą ilością metod identyfikacji twarzy, np. AWS Rekognition lub algorytmy dostępne na platformie Github w celu prostszego porównania ich działania.

