\section{Detekcja}
\subsection{Zdjęcie na wprost}
\subsection{Zdjęcie częściowo od boku}
\subsection{Przechylona głowa}
\subsection{Częściowo zakryta twarz}
\subsection{Wnioski}

\section{Dane treningowe}
Wybierz jakas baze, delikatnie opisz i ile probek bierzesz per osoba
teraz mam: thumbnails features deduped sample

\section{Trenowanie sieci neuronowych}
dodać czas w pythonie
\subsection{Wpływ ilości próbek na czas tworzenia modelu}
kilka testów
\subsection{Wpływ ilości próbek na rozmiar modelu}
kilka testów

\section{Rozpoznawanie twarzy}
\subsection{Wpływ ilości próbek na pewność rozpoznania}

\subsection{Porównanie wyników}
\subsection{Czas przetwarzania zapytania}

\section{Ocena przydatności wybranych usług IoT}



Konfiguracja bazy danych oraz maszyny wirtualnej okazała się równie prosta w każdym ze środowisk. Proces konfiguracji odbywał się poprzez wypełnienie kilku formularzy niewymagających wprowadzania dużej ilości informacji (ze względu na ograniczenia studenckiej licencji). Na końcu procesu uzyskany zostaje connection string oraz konto za pomocą, którego można zalogować się na serwer.

Największa różnica między dwoma dostawcami wystąpiła w przypadku usług hostujących aplikację webową czyli Azure App Service oraz AWS Elastic Beanstalk. Podczas pierwszych testów konfiguracji aplikacja internetowa istniała jedynie w rozwiązaniu przygotowanym w języku Angular 4. Struktura aplikacji została przygotowana na podstawie wzoru przygotowanego przez Microsoft. Azure App Service bezproblemowo wspierał nawet najnowsze wersje rozwiązań przygotowanych dla .NET Core 2. Niestety AWS nie był przygotowany 