\section{Budowa aplikacji}
Aplikację, którą utworzono na potrzeby tej pracy można podzielić na 3 moduły przedstawione na rysunku \ref{fig:schemat_systemu}. Użyte moduły to:
\begin{itemize}
\item Aplikacja webowa- interfejs pozwalający na zlecanie nowych zadań aplikacji konsolowej i odczyt wyników przesłanych przez obie aplikacje konsolowe,
\item Aplikacja konsolowa- przetwarza zadania detekcji oraz rozpoznawania twarzy zlecone za pomocą aplikacji webowej, trenuje algorytmy wybranym zestawem danych,
\item Aplikacja konsolowa zarządzająca domem- przekazuje cyklicznie odczytywane dane z czujników i kamery.
\end{itemize}
\begin{figure}[H]
	\centering
	\includegraphics[scale=0.7]{schemat_systemu.png}
	\caption{Budowa systemu zarządzania domem}
	\label{fig:schemat_systemu}
\end{figure}
We wczesnej fazie projektu istniała jedna aplikacja konsolowa ale ze względu na złożoność programu przetwarzającego zadania została ona rozdzielona na aplikację, która musi być uruchamiana na Raspberry Pi (program zarządzający domem) oraz drugą, którą można uruchomić w dowolnym innym środowisku. Wykorzystane usługi pomocnicze to:
\begin{itemize}
\item baza danych MSSql- przechowywanie danych o zadaniach, wynikach, nauczonych sieciach oraz profilach,
\item Dropbox- przechowywanie obrazów oraz nauczonych modeli sieci,
\item Azure Cognitive Services- usługa przetwarzająca obrazy (rozdział \ref{azurecs}).
\end{itemize}

\section{Konfiguracje uruchomieniowe}
W celu zmaksymalizowania wydajności modułów, podjęto decyzję o uruchomieniu każdego z nich w odrębnym środowisku. System został uruchomiony w dwóch konfiguracjach. Niezależnie dla konfiguracji niezmienny pozostał moduł aplikacji konsolowej zarządzającej domem, który ze względu na wymaganie fizycznego dostępu do urządzeń podłączonych do Raspberry Pi zawsze był uruchamiany w tym środowisku.
Pierwsza z konfiguracji została oparta o środowisku chmurowe Azure, a druga AWS. Usługi dedykowane dla poszczególnych modułów przedstawiono na rysunku \ref{fig:schemat_azure} i \ref{fig:schemat_aws}.

\begin{figure}[H]
	\centering
	\includegraphics[scale=0.5]{budowa_azure.png}
	\caption{Podział systemu na usługi Azure}
	\label{fig:schemat_azure}
\end{figure}

\begin{figure}[H]
	\centering
	\includegraphics[scale=0.5]{budowa_aws.png}
	\caption{Podział systemu na usługi AWS}
	\label{fig:schemat_aws}
\end{figure}

