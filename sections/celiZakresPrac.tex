Celem pracy było stworzenie systemu bezpieczeństwa i zarządzania domem w IoT. Głównymi założeniami było opracowanie systemu pozwalającego na porównanie wybranych metod i usług pozwalających na analizę obrazu oraz kontrolę stanu czujników podłączonych do systemu. IoT jest bardzo prężnie rozwijającą się ideą, dlatego postanowiono porównać wybrane technologie i usługi ułatwiające zarządzanie oraz integrację z takimi aplikacjami.

Jako platformę sprzętową wybrano bardzo popularne w zastosowaniach IoT urządzenie Raspberry Pi 3. Z powodu jego ograniczonych zasobów obliczeniowych zintegrowano chmurowe usługi Azure oraz AWS. Zakres pracy obejmuje następujące zagadnienia:
\begin{itemize}
\item dobór odpowiedniej platformy sprzętowej oraz oprogramowania,
\item porównanie działania wybranych algorytmów detekcji oraz rozpoznawania twarzy,
\item wybór i porównanie wybranych technologi umożliwiających integrację z systemami IoT,
\item wykrywanie zdarzeń rejestrowanych przez wybrane czujniki
%TODO is it really needed?
\item opracowanie systemu przesyłania powiadomień
\item porównanie przydatności zintegrowanych usług w zastosowaniu dla systemu bezpieczeństwa domu
\end{itemize}