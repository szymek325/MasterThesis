Mikrokontroler to system mikroprocesorowy zrealizowany w postaci pojedynczego układu scalonego zawierającego jednostkę centralną (CPU), pamięć RAM, rozbudowane układy wejścia-wyjścia oraz przeważnie pamięć programu FRAM, MRAM, ROM lub Flash. Pierwszym seryjnie produkowanym mikrokontrolerem był układ Intel 8048, sprzedawany od 1976 roku. Wynalazcą mikrokontrolera był Gary Boone z firmy Texas Instruments.

Określenie mikrokontroler pochodzi od głównego obszaru jego zastosowań, jakim jest sterowanie urządzeniami elektronicznymi, takimi jak: urządzenia biurowe, medyczne, zdalnego sterowanie, systemy sterowania silnikami samochodowymi oraz zabawki i inne systemy wbudowane. Mikrokontroler stanowi użyteczny i całkowicie autonomiczny system mikroprocesorowy, nie wymagający użycia dodatkowych elementów, których wymagałby do pracy tradycyjny mikroprocesor. Mikrokontrolery przystosowane są do bezpośredniej współpracy z rozmaitymi urządzeniami zewnętrznymi.

Wraz z rozwojem człowieka, następowało coraz większe zapotrzebowanie na rozwinięcie metod grzewczych i chłodniczych, znajdujących zastosowanie w życiu codziennym. Proces ten rozpoczął się od powstania lodówek i piekarników, aż następnie rozwinął się w zaawansowane systemy chłodnicze i grzewcze, dostępne dla każdego człowieka. Rozwój tych technologii był napędzany w szczególności przez duże zapotrzebowanie w przemyśle, np. przy obróbce różnego rodzaju materiałów i do przechowywania produktów wrażliwych na zbyt wysoką temperaturę.

Ogrzewanie to proces polegający na dostarczeniu energii termicznej do pewnego ciała lub pomieszczenia, w celu podniesienia temperatury. Chłodzenie to proces polegający na odprowadzeniu energii termicznej z układy, w celu uzyskania niższej temperatury.

Powstanie pierwszego regulatora temperatury, przypisuje się Korneliuszowi Drebbelowi, który stworzył pierwsze urządzenie ze sprzężeniem zwrotnym, w 1620 roku. System automatycznie sterował temperaturą pieca przemysłowego. Następnie regulatory temperatury były wykorzystywane w inkubatorach do wykluwania piskląt. Pierwszy regulator, który znalazł zastosowanie w przemyśle, powstał w 1777 roku. Został wykorzystany w piecu ciepłowni dostarczającej ciepłą wodę.
Przez ostatnie lata, regulatory zostały bardzo mocno rozwinięte i są wykorzystywane w prawie każdej dziedzinie przemysłu. Przekrój dostępnych rozwiązań jest bardzo szeroki.

W rozdziale drugim pt. ,,Cel i zakres pracy'' przedstawiono główne założenia projektu oraz zakres prac autora.

W kolejnym rozdziale pt. ,,Podstawy teoretyczne'' można znaleźć opis głównych urządzeń wykonawczych, o które została oparta praca. Ukazano tam m.in. opis płytki Arduino, ogniwa Peltiera oraz podstawowe informacje na temat technologii internetowych.

Rozdział czwarty pt. ,,Projekt układu pomiarowo-wykonawczego'' przedstawia etap projektowania i konstrukcji obiektu oraz układu pomiarowego.

W rozdziale piątym pt. ,,Arduino'' opisano zastosowane rozwiązania oraz działanie kodu programu sterującego temperaturą.

Rozdział szósty pt. ,,Aplikacja mobilna'' główne założenia dotyczące aplikacji, jej konstrukcję i sposób działania.

Rozdział siódmy i ósmy przedstawia przebieg przeprowadzonych testów oraz wyniki. Przedstawione zostały wnioski w odniesieniu do przyjętych założeń projektowych.