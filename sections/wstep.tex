Internet rzeczy (Internet of Things -IoT) to koncepcja przedstawiająca sieć urządzeń fizycznych połączonych inteligentną siecią KNX lub internetową, komunikujących się między sobą i wymieniających dane. Podstawowym celem IoT jest stworzenie inteligentnych przestrzeni- miast, budynków lub systemów związanych z życiem codziennym. Jednym z zastosowań takich systemów są inteligentne domy (smart homes), wykorzystujące czujniki oraz aktuatory do zarządzania domem.

Na potrzeby tej pracy opracowano system bezpieczeństwa oparty o system detekcji i rozpoznawania twarzy na obrazie.

%TODO dodać coś o przetwarzani obrazu, rozpoznawaniu twarzy itd

W rozdziale ,,Cel i zakres pracy'' przedstawiono główne założenia projektu oraz zakres prac autora.

W rozdziale ,,Wstęp teoretyczny'' omówiono podstawowe zagadnienia związane z treścią tej pracy magisterskiej.

W rozdziale ,,Przegląd dostępnych metod rozpoznawania twarzy'' porównano możliwości kilku z dostępnych usług lub oprogramowania pozwalającego na rozpoznawanie twarzy.

W rozdziale ,,System zarządzania metodami rozpoznawania twarzy'' przedstawiono strukturę stworzonej aplikacji, jej podział na moduły oraz wybrane dla każdego z nich środowisko uruchomieniowe.

W rozdziale ,,Aplikacja internetowa'' zaprezentowano wszystkie strony utworzone na potrzeby projektu.

W rozdziale ,,Aplikacja konsolowa'' omówiono sposób wykorzystania wcześniej opisanych usług oraz algorytmy odpowiedzialne za działanie systemu do testowania różnych rozwiązań dla problemu identyfikacji twarzy.

W rozdziale ,,Aplikacja konsolowa do zarządzania domem'' opisano sposób wykorzystania czujników oraz kamery podłączonej bezpośrednio do platformy Raspberry Pi.

W rozdziale ,,Porównanie wykorzystanych technologi'' zbadano oraz porównano skuteczność działania technologi wybranych na potrzeby tej pracy.

W rozdziale ,,Podsumowanie'' zawarto podsumowanie zgromadzonych informacji oraz przedstawiono wnioski wynikające z przeprowadzonych badań.



%TODO dodać opisy kolejnych rozdziałów
